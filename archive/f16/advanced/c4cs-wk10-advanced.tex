\documentclass{article}
\usepackage{amssymb}
\usepackage{comment}
\usepackage{courier}
\usepackage{fancyhdr}
\usepackage{fancyvrb}
\usepackage[T1]{fontenc}
\usepackage[top=.75in, bottom=.75in, left=.75in,right=.75in]{geometry}
\usepackage{graphicx}
\usepackage{lastpage}
\usepackage{listings}
\lstset{basicstyle=\small\ttfamily}
\usepackage{mdframed}
\usepackage{parskip}
\usepackage{ragged2e}
\usepackage{soul}
\usepackage{upquote}
\usepackage{xcolor}

% http://www.monperrus.net/martin/copy-pastable-ascii-characters-with-pdftex-pdflatex
\lstset{
upquote=true,
columns=fullflexible,
literate={*}{{\char42}}1
         {-}{{\char45}}1
         {^}{{\char94}}1
}

% http://tex.stackexchange.com/questions/40863/parskip-inserts-extra-space-after-floats-and-listings
\lstset{aboveskip=6pt plus 2pt minus 2pt, belowskip=-4pt plus 2pt minus 2pt}

\usepackage[colorlinks,urlcolor={blue}]{hyperref}

\begin{document}

\fancyfoot[L]{\color{gray} C4CS -- F'16}
\fancyfoot[R]{\color{gray} Revision 1.2}
\fancyfoot[C]{\color{gray} \thepage~/~\pageref*{LastPage}}
\pagestyle{fancyplain}


\title{\textbf{Advanced Excercise 10\\}}
\author{\textbf{\color{red}{Due: Saturday, December 3, 10:00PM (Hard Deadline)}}}
\date{}
\maketitle


\section*{Submission Instructions}
To receive credit for this assignment you will need to stop by someone's
office hours and demo your local AFS access.

\emph{You are welcome to do this assignment either on your native operating
  system on your machine or inside the class VM, whichever you are more
  comfortable with. I have done this on Linux and OS X. I have not personally
  tried this on Windows, however I believe it should work.}

\section*{The Andrew File System}

One thing you've hopefully noticed is that when you log on to any CAEN machine
(or ssh to any CAEN login server) all of your files are magically there.
That's because your files are stored using AFS, a network file system. When
you save files on a CAEN machine, they don't actually save to the local hard
disk, instead they go out over the Internet to a machine in a server room
somewhere.\footnote{Last I checked, for folks with uniqnames starting a-k,
this was the MITC Data Center on Oakbrook off of State. I don't recall where
the rest of the alphabet is stored. I think it was in the Southfield data
center at one point.} File reads similarly go over the network to grab the
data, so everything stays in sync automatically.\footnote{This glosses over
  some details about caching that's done for performance, but the basic idea
  is there.} Note that while this is similar to Dropbox, it is slightly
different as well since Dropbox stores and edits local copies on disk and then
syncs them in the background. That wouldn't work at the university scale,
imagine if every CAEN machine had to have enough disk space to hold every file
for every student! This means that AFS will only work if you are online.

Since AFS is a networked file system, you can also set up your local machine
to join the network.  This way, whenever you change anything on your machine,
it'll automatically update on CAEN and vice-versa. You could run your text
editor on your machine and your compiler on CAEN if you like. Along the way,
you'll also be able to set things up so that you don't need to type your
password every time that you ssh into CAEN.


\subsection*{Kerberos}

To get things working, there are two key things you need to get set up, the
first is \textbf{Kerberos}. Kerberos is a tool for authentication. You've been
using it every time you log into CAEN (or Cosign\footnote{
  You can actually set up your web browser to not need your password either.
  It can also use the same Kerberos ticket. I know this is do-able in Firefox,
  it may be do-able in other browsers as well.
}). With Kerberos, you must periodically (every 24~hours in umich's setup)
enter your password to get a new \emph{ticket}. Once you have this ticket on
your machine, however, you can use it to log into any Michigan service without
ever typing your password again. AFS uses this ticket to gain access to the
files that you own.

\emph{Big Hint:} You will need a configuration file (\texttt{/etc/krb5.conf}).
Don't try to write this yourself! Instead login to a CAEN machine and grab a
copy from that machine.

To test if you have Kerberos working, you can add this to your
\texttt{.ssh/config} file on your local machine:

\begin{lstlisting}
  Host login.engin.umich.edu
  GSSAPIAuthentication yes
  GSSAPIDelegateCredentials yes
  GSSAPITrustDNS yes
\end{lstlisting}

If you have a valid Kerberos ticket, when you run
\texttt{ssh~login.engin.umich.edu} you should not be prompted for a password,
you'll just log straight in!

\emph{Note:} If you are on macOS Sierra, Apple removed support for the
\texttt{GSSAPITrustDNS} option. You can still test your Kerberos setup by
connecting to a CAEN server directly (no modifications to\texttt{~/.ssh/config}
necessary):

\begin{lstlisting}
  ssh -K uniqname@caen-vnc-vm14.engin.umich.edu
\end{lstlisting}

\subsection*{AFS}

The next thing you'll need to install and set up is AFS. The same hint applies
here where the best way to find the correct AFS configuration is look at it on
a CAEN machine and then copy it to your local machine.

Notice that once you install AFS you will have \emph{list} permissions on much
of the directory hierarchy. This will let you explore a little, such as seeing
that files exist, but not their contents. For example, while holding
credentials for \texttt{ppannuto}, I try to look at Matt's files, I can see
that he has a \texttt{test.sh} file in his home directory, but not what's in
it:

\begin{lstlisting}
  $ ls /afs/umich.edu/user/m/t/mterwil
  ls: cannot access /afs/umich.edu/user/m/t/mterwil/Private: Permission denied
  ls: cannot access /afs/umich.edu/user/m/t/mterwil/MPrint: Permission denied
  ls: cannot access /afs/umich.edu/user/m/t/mterwil/Desktop: Permission denied
  ls: cannot access /afs/umich.edu/user/m/t/mterwil/Downloads: Permission denied
  ls: cannot access /afs/umich.edu/user/m/t/mterwil/bin: Permission denied
  ls: cannot access /afs/umich.edu/user/m/t/mterwil/Documents: Permission denied
  ls: cannot access /afs/umich.edu/user/m/t/mterwil/Music: Permission denied
  ls: cannot access /afs/umich.edu/user/m/t/mterwil/Pictures: Permission denied
  ls: cannot access /afs/umich.edu/user/m/t/mterwil/Videos: Permission denied
  ls: cannot access /afs/umich.edu/user/m/t/mterwil/test.sh: Permission denied
  bin  Desktop  Documents  Downloads  MPrint  Music  Pictures  Private  Public Shared  test.sh  Videos
  $ cat /afs/umich.edu/user/m/t/mterwil/test.sh
  cat: /afs/umich.edu/user/m/t/mterwil/test.sh: Permission denied
\end{lstlisting}

If you find yourself getting a similar set of permission denied issues in your
own home directory, it means you are connected to AFS correctly, but you
haven't connected Kerberos and AFS correctly.


\subsection*{The Assignment}

\textbf{Get AFS installed and working on your local machine. You should have
read/write access to your own files.}

The writeup in this assignment gives you some intuition for what's going on
and the steps you'll have to take, but it is not a step-by-step guide -- there
are plenty of those out there (I even found a few umich-specific ones for the
Kerberos half, not that there's anything special you have to do to set things
up to work here, but they use umich as their example).

\end{document}
