\documentclass{article}
\usepackage{amssymb}
\usepackage{courier}
\usepackage{fancyhdr}
\usepackage[top=.75in, bottom=.75in, left=.75in,right=.75in]{geometry}
\usepackage{graphicx}
\usepackage{lastpage}
\usepackage{listings}
\lstset{basicstyle=\small\ttfamily}
\usepackage{soul}
\usepackage{tabularx}
\usepackage{xcolor}

\usepackage[colorlinks,urlcolor={blue}]{hyperref}

\begin{document}

\fancyfoot[L]{\color{gray} C4CS -- W'16}
\fancyfoot[R]{\color{gray} Revision 1.0}
\fancyfoot[C]{\color{gray} \thepage~/~\pageref*{LastPage}}
\pagestyle{fancyplain}


\title{\textbf{Homework 3\\Editors}}
\author{Assigned: Friday, January 22, 11:00AM}
\date{\textbf{\color{red}{Due: Friday, January 29, 11:00AM (Hard Deadline)}}}
\maketitle


\section*{Submission Instructions}
Submit this assignment on \href{https://gradescope.com/courses/2248}{Gradescope}.
You must submit every page of this PDF.
We recommend using the free online tool \href{https://www.pdfescape.com}{PDFescape}
to edit and fill out this PDF.
You may also print, handwrite, and scan this assignment.

\medskip
\noindent
\hl{Even if there are not questions to answer on the first page, please still
submit every page of this PDF.}

\medskip
\noindent
There may multiple answers for each question. If you are unsure,
state your assumptions and your reasoning for why you think your answer
makes sense.

\section*{Readings}
\emph{These readings are not required, not graded, and I will not ask any
questions on them. They are simply articles that I think may be worth your time.}

{\small

\subsection*{Interrupting Developers and Makers vs Managers}
In lecture we spent a little time talking about the mental model of
programmers. As computer science grows, there is a growing body of research
and wisdom in how to maximize the efficacy of individual programmers.
One of the key revelations is that programming is a creative process. It
requires time to ``page in'' what you are working on. As a result, even small
interruptions (read: text messages) can be far more costly than you may
realize.

\medskip
\noindent
\url{http://thetomorrowlab.com/2015/01/why-developers-hate-being-interrupted/}\\
\emph{Why developers hate being interrupted}, by Derek Johnson at The Tomorrow Lab.

\noindent
\url{http://www.paulgraham.com/makersschedule.html}\\
\emph{Maker's Schedule, Manager's Schedule}, by Paul Graham, co-founder of Y~Combinator.

\medskip
\noindent
Once you are in a job environment and meetings become a regular thing, one of
the best tricks I learned is to schedule a few 4-hour meetings with yourself
throughout the week. It both prevents others from interrupting you (your
calendar shows you as busy) and lets you mentally prepare for a good, reliable
work session.
(A similar alternative, some find
\href{http://lifehacker.com/productivity-101-a-primer-to-the-pomodoro-technique-1598992730}{pomodoro}
helpful)

%As you work on homework this week, especially for your programming-heavy EECS
%classes, I encourage you to try turning off your phone (yes, seriously, off),
%close your browser and start from a blank session (hiding the browser window
%with your Facebook tab doesn't help if messenger keeps dinging), and see if
%your software development experience is any different.\footnote{
%  It can be hard to be not keep opening a Facebook tab every time you Google
%  something. Sometimes when I am having a particularly hard time forcing
%  myself to work, I'll ``black hole'' things I might be tempted to distract
%  myself with: \\
%  \texttt{printf "\# Disable some sites\char`\\n127.0.0.1\char`\\tfacebook.com\char`\\n127.0.0.1\char`\\treddit.com\char`\\n" | sudo tee -a /etc/hosts}\\
%  You can undo this by editing \texttt{/etc/hosts} later.
%}

\subsection*{Software Engineering in the Real World}
\url{http://blogs.msdn.com/b/peterhal/archive/2006/01/04/509302.aspx}\\
\emph{What Do Programmers Really Do Anyway?}, by Peter Hallam, then-Microsoft~Developer
\begin{quote}
  \emph{%
    Why is 5 times more time spent modifying code than writing new code? The
    answer is that new code becomes old code almost instantly.  Write some new
    code. Go for coffee. All of sudden you've got old code.
  }
\end{quote}
This post gives a nice perspective on what enterprise software engineering is
really like. School projects are misleading. Rarely in life will you be faced
with a spec and a blank text file. Far more often you are building on or
improving something that came before you and will exist long after you.

\medskip
\noindent
\url{http://www.joelonsoftware.com/articles/fog0000000069.html}\\
\emph{Things You Should Never Do, Part I}, by Joel Spolsky
\begin{quote}
  \emph{It's harder to read code than to write it.}
\end{quote}
Building on the previous, rarely in life are you handed a spec and a blank text
file, often in life you are handed a spec and a pile of (seemingly) spaghetti
code that does at least some of it. This article discusses the hidden value of
the built-up institutional knowledge and the high value of working code.

\medskip
\noindent
\emph{When was the last time you saw a hunt-and-peck pianist?} -Jeff Atwood,
co-founder of Stack Overflow and Discourse.\\
Some extras on the
\href{http://blog.codinghorror.com/we-are-typists-first-programmers-second/}
{value of being a good typist}
and
\href{http://blog.codinghorror.com/going-commando-put-down-the-mouse/}
{ditching the mouse for the keyboard}.

}

% I had some intention of some articles on text editors here, but I think these
% are actually more useful, if a bit preachy


\newpage
\section{The Basics}

To kick off lecture, we talked about one of the earliest text editors,
\texttt{ed}. While \texttt{ed} is rarely used today, its cousin
\texttt{sed} (``stream editor''), can often be a useful tool for
doing quick, simple (or remarkably complex) substitutions on files.

\medskip
\noindent
\texttt{sed} is a pretty powerful tool, with a lot of features. By far its most
used feature, however, is ``s commands''.

\medskip
\noindent
\textbf{Below is a transcript of a terminal session. Fill in the blanks to
generate the output as shown:}


\begin{lstlisting}[mathescape=true]
[1]: cp input.txt backup.txt
[2]: cat input.txt
one two three
one two three three two one
One more here
What about ONE oNe oneone oneoneone oNeOnEoNe?

[3]: sed $\rule[-1mm]{6cm}{0.10mm}$ input.txt
zero two three
zero two three three two one
One more here
What about ONE oNe zeroone oneoneone oNeOnEoNe?

[4]: cat input.txt  # Notice that the input file hasn't changed
one two three
one two three three two one
One more here
What about ONE oNe oneone oneoneone oNeOnEoNe?

[5]: sed $\rule[-1mm]{6cm}{0.10mm}$ input.txt
zero two three
zero two three three two one
zero more here
What about zero oNe oneone oneoneone oNeOnEoNe?

[6]: sed $\rule[-1mm]{6cm}{0.10mm}$ input.txt
zero two three
zero two three three two zero
zero more here
What about zero zero zerozero zerozerozero zerozerozero?

[7]: sed $\rule[-1mm]{6cm}{0.10mm}$ input.txt  # Careful, two things change here
[8]: cat input.txt
zero two three
zero two three three two zero
One more here
What about ONE oNe zerozero zerozerozero oNeOnEoNe?
\end{lstlisting}


\subsection*{Explaining a very annoying ``gotcha''}

Instead of command \texttt{[7]}, you might be tempted to write something like:
\begin{quote}
  \texttt{sed [...]\ input.txt > input.txt}
\end{quote}
However, you will be disappointed by the outcome. Let's play with \texttt{cat}
because it's a little easier:

\begin{lstlisting}
[1]: echo one > 1 ; echo two > 2
[2]: cat 1 2 | cat > 3
[3]: cat 1 2 | cat > 1
\end{lstlisting}

\noindent
\textbf{What is the contents of 3? What is the contents of 1? Give a
reasonable guess for why the contents of 1 and 3 are different.}



\newpage


\section{Text-based editors (Basics)}

By default, Ubuntu ships with a minimal \texttt{vim} and no \texttt{Emacs}. If
you haven't already, start by installing each:
\begin{quote}
  \texttt{sudo apt-get install vim-gnome emacs}
\end{quote}
%
One recurring theme in this class is ``neither is better''.
For text editors, this means you invariably will encounter both \texttt{vim}
and \texttt{Emacs} environments in your career. While there is little benefit
in becoming an expert in both, it's very valuable to have basic skills in
the most common editors as you will encounter them.

This question simply asks you to list how to do what I consider the minimum
skills in each editor. While I cannot force you to open a text editor and try
each of these to build some muscle memory, I can say that having these basics
at your fingertips is good, and in a setting such as a final exam, it would be
perfectly reasonable to expect you to recall all of these correctly.

\begin{quote}
Just starting out with these? Never used one before? Try the built-in tutorial
programs. Run the program \texttt{vimtutor} for vim, or open Emacs and then
hit \texttt{Ctrl-h} followed by \texttt{t} for Emacs.
\end{quote}
%
For each of the following, assume the editor was just opened (e.g.\ you have
just typed \texttt{vim hello.c}).

\medskip
\medskip
\begin{tabularx}{\textwidth}{X|X}
  vim                               & Emacs                              \\
  \hline
                                    &                                    \\
  Save file                         & Save file                          \\
                                    &                                    \\
                                    &                                    \\
                                    &                                    \\
                                    &                                    \\
                                    &                                    \\
  Quit \textbf{without} saving edits& Quit \textbf{without} saving edits \\
                                    &                                    \\
                                    &                                    \\
                                    &                                    \\
                                    &                                    \\
                                    &                                    \\
  Save and quit (two commands fine) & Save and quit (two commands fine)  \\
                                    &                                    \\
                                    &                                    \\
                                    &                                    \\
                                    &                                    \\
                                    &                                    \\
                                    &                                    \\
  Add the text ``Hello World''      & Add the text ``Hello World''       \\
  at the current location           & at the current location            \\
                                    &                                    \\
                                    &                                    \\
                                    &                                    \\
                                    &                                    \\
                                    &                                    \\
  Find the text ``TODO''            & Find the text ``TODO''             \\
                                    &                                    \\
                                    &                                    \\
                                    &                                    \\
                                    &                                    \\
                                    &                                    \\
  Find the next ``TODO''            & Find the next ``TODO''             \\
                                    &                                    \\
                                    &                                    \\
                                    &                                    \\
                                    &                                    \\
                                    &                                    \\
\end{tabularx}



\newpage
\section{Text-based editors (The Fun Stuff)}

People often complain about how difficult the simple things are using
text-based text editors. Indeed, Notepad can do everything from the previous
question (and Notepad is much easier to use). In this question we explore some
of the things that Notepad (and nano, and gedit) cannot do.

\medskip
\noindent
I encourage you to play around with things as you work through this question.
The intent is to expose you to features you may not have been aware of along
with some context for why they are useful.

\medskip
\noindent
There are many things to try here.
\textbf{For credit on the homework, you only need to choose any 5.}
That said, I encourage you to try all of these.

\medskip
\noindent
For this question, you may choose \texttt{vim} or \texttt{Emacs}, whichever
you are more comfortable with.

\medskip
\noindent
To start, grab copies of two files full of code:
\begin{quote}
  \texttt{\# Examples from \url{http://web.mst.edu/~price/cs53/code_example.html}}\\
  \texttt{wget \url{http://web.mst.edu/~price/cs53/fs11/workDecider.cpp}}\\
  \texttt{wget \url{http://web.mst.edu/~price/cs53/KatsBadcode.cpp}}
\end{quote}


\begin{enumerate}
  \item Sometimes you will come across some ugly code. Your editor can help
    make it better. Open \texttt{KatsBadcode.cpp}. Among other issues, the
    really bad tabbing makes this hard to read.\\
    \textbf{Describe how to automatically fix the whitespace for the whole file.}
    \vspace{4em}
  \item Editing one file is nice, but often it's really useful to compare
    files side-by-side (source and headers, spec and implementation).\\
    \textbf{Describe the command sequence to create another window side-by-side with your current window, switch to it, and then open a file in that window.}\\
    \vspace{4em}

    Try playing around with these two views, copy/paste code between them,
    bind them so they scroll together, resize them, add more splits.\\
  \item Editors also have pretty nice integration with compilation tools. With
    \texttt{KatsBadcode.cpp} open, try typing \texttt{:make KatsBadcode} in
    \texttt{vim} or \texttt{M-x compile<enter>make KatsBadcode} in
    \texttt{Emacs}.\footnote{
      Descriptions of \texttt{Emacs} commands are usually written as
      \texttt{C-c} or \texttt{M-x}, which mean ``Ctrl+C'' or ``Meta+x''
      respectively. Meta is often mapped to the escape and/or alt key,
      \texttt{M-x} means press Escape and then x or hold alt and press x.
    }
    \begin{quote}
      First cool thing that happened: Yes, you can run \texttt{make}
      \emph{without} any Makefile anywhere. We'll cover how that happened
      during build-system week.
    \end{quote}
    Building \texttt{KatsBadcode.cpp} will fail with several errors.\\
    \textbf{Describe how to navigate between compilation errors automatically.}
    \vspace{4em}
  \item While C-style languages support \texttt{/* block comments */}, others
    such as Python have no block comments and require you to put a \texttt{\#}
    at the beginning of every line.\\
    \textbf{Describe how to efficiently comment out a large block of Python code.}
    \vspace{4em}
  \item (This one is actually about the terminal emulator): The normal
    keyboard shortcuts to copy/paste are Ctrl-C and Ctrl-V. These do not work
    in the terminal, however.\\
    \textbf{Explain what Ctrl-C does in a terminal.}\\
    ~\\
    ~\\
    ~\\
    ~\\
    \textbf{Describe how to copy/paste using the keyboard in a terminal.}
    \vspace{4em}
  \item The default cut/copy/paste(/put/yank/kill) operations do not put text in
    the system clipboard (that is you cannot copy/paste into/from other
    applications).
    \textbf{Describe how to copy/paste a region of text to/from the system
    clipboard using your text editor.}
    \vspace{4em}
  \item Many times you have a repetitive task that you need to do say 10 or 20
    times. It's too short to justify writing a script or anything fancy, but
    it's annoying to type the same thing over and over again. As example,
    reformatting the staff list to a nice JSON object:
    \begin{verbatim}                                staff = [
Pannuto,Pat                        {"first": "Pat", "last": "Pannuto"},
Darden,Marcus                      {"first": "Marcus", "last": "Darden"},
Smith,Max                          {"first": "Max", "last": "Smith"},
Snider,David                       {"first": "David", "last": "Snider"},
Khan,Waleed                        {"first": "Waleed", "last": "Khan"},
Terwilliger,Matt                   {"first": "Matt", "last": "Terwilliger"},
Chojnacki,Alex                     {"first": "Alex", "last": "Chojnacki"},
Hussein,Mo                         {"first": "Mo", "last": "Hussein"}
                                ]\end{verbatim}
    Editors support the record and reply of \emph{macros}. With macros, you
    can start recording, puzzle out how to turn one line into the other, and
    then simply replay it for the rest. As a general rule, do not bother
    trying to be optimal or efficient, just find anything that works.
    Try starting with the column on the left and devising a macro to turn it
    into the column on the right.\\
    \textbf{Describe how to record and replay a macro.}
    \emph{\small (you do not need to include the contents of your macro)}
    \vspace{4em}
    % Easter egg for those who look, here is mine (in vim):
    % yypws^M^[i{"first": ^[A,^[kddpkJli"last": ^[A}^[0f:lli"^[wwi"^[f:lli"^[$i"^[kJ0j
\newpage
  \item Similar to how \texttt{$\sim$/.bashrc} configures your shell, a
    \texttt{$\sim$/.vimrc} or \texttt{$\sim$/.emacs} file\footnote{
      These files do not exist by default, you have to create them.
    } can configure how your editor behaves. Here are some excerpts from the
    staff's configuration files:
    \begin{lstlisting}
    .vimrc:
    set number    " double-quote means comment in vimscript; this turns on line numbers
    map ; :       " this line and the next makes ; act like : so that you don't have to
    noremap ;; ;  " hit shift all the time, typing ";;" will act as ";" used to

    .emacs:
    ; line numbers - in Emacs, ; means a comment
    (global-linum-mode t)
    (setq linum-format "%d ")
    ; Changes all yes/no questions to y/n type
    (fset 'yes-or-no-p 'y-or-n-p)
    \end{lstlisting}
    \textbf{Add something useful not listed above to your editor's configuration file.\\
      Describe  what you added and why.}
    \vspace{12em}
  \item Some final little things
    \small
    \begin{enumerate}
      \item Editors understand balanced ()'s and \{\}'s.
        \textbf{Describe how to jump from one token to its match, such as from
          the opening \{ on line~29 of \texttt{KatsBadcode} to its partner \}
          on line~58.
        }
        \vspace{4em}
      \item Now imagine if you are editing code and you determine you can
        remove an if block, for example removing the \texttt{if} on line~28 of
        \texttt{KatsBadcode} and running its body unconditionally. You need to
        fix the indentation level of all 20~lines of the body.\\
        \textbf{Describe how to change the indentation level of a block of code.}
        \vspace{4em}
      \item Sometimes it's useful to run quick little commands. For example,
        your code needs to read from a file, but you can't remember if it's
        called \texttt{sample\_data.txt} or \texttt{sampleData.txt}.\\
        \textbf{Describe how to run `ls' directly from within your editor.}
        \vspace{4em}
    \end{enumerate}
\end{enumerate}

%\newpage
%\section{GUI-based editors}
% This has gotten long; we'll save this for the IDE week


\end{document}
