\documentclass{article}
\usepackage[T1]{fontenc}

\usepackage{amssymb}
\usepackage{courier}
\usepackage{fancyhdr}
\usepackage{fancyvrb}
\usepackage[top=.75in, bottom=.75in, left=.75in,right=.75in]{geometry}
\usepackage{graphicx}
\usepackage{lastpage}
\usepackage{listings}
\lstset{basicstyle=\small\ttfamily}
\usepackage{mdframed}
\usepackage{parskip}
\usepackage{soul}
\usepackage{tabularx}
\usepackage{textcomp}
\usepackage{upquote}
\usepackage{xcolor}

% http://www.monperrus.net/martin/copy-pastable-ascii-characters-with-pdftex-pdflatex
\lstset{
upquote=true,
columns=fullflexible,
keepspaces=true,
literate={*}{{\char42}}1
         {-}{{\char45}}1
         {^}{{\char94}}1
}
%\lstset{
%  moredelim=**[is][\color{blue}\bf\small\ttfamily]{@}{@},
%}

% http://tex.stackexchange.com/questions/40863/parskip-inserts-extra-space-after-floats-and-listings
\lstset{aboveskip=6pt plus 2pt minus 2pt, belowskip=-4pt plus 2pt minus 2pt}



\usepackage[colorlinks,urlcolor={blue}]{hyperref}
\usepackage[capitalise,nameinlink,noabbrev]{cleveref}
\crefname{section}{Question}{Questions}
\Crefname{section}{Question}{Questions}

\begin{document}


\fancyhead[C]{\hl{Select the right page in Gradescope or we will not grade the question!}}
\fancyhead[L]{}
\fancyhead[R]{}

\fancyfoot[L]{\color{gray} C4CS -- F'17}
\fancyfoot[R]{\color{gray} Revision 1.0}
\fancyfoot[C]{\color{gray} \thepage~/~\pageref*{LastPage}}
\pagestyle{fancyplain}

\title{\textbf{Homework 3\\Shells, Environment, and Scripting}}
\author{\textbf{\color{red}{Due: Wednesday, October 4th, 11:59PM (Hard Deadline)}}}
\date{}
\maketitle


\section*{Submission Instructions}
Submit this assignment on \href{https://gradescope.com/courses/9999}{Gradescope}.
You may find the free online tool \href{https://www.pdfescape.com}{PDFescape}
helpful to edit and fill out this PDF.
You may also print, handwrite, and scan this assignment.


\section{Understanding your \texttt{PATH}}

In a terminal, type \texttt{PATH=} (just hit enter after the equal sign, no
space characters anywhere). Try to use the terminal like normal (try running
\texttt{ls}). What happened?

\textbf{Give an example of a command that used to work but now doesn't:}
\begin{quote}
  \color{violet}
  Most things don't work \texttt{mv, cp, rename, gcc, make, ...}. I even gave
  you one example in the question itself (\texttt{ls}).
\end{quote}

\textbf{Can you still run this command with an empty \texttt{PATH}? How?}
\begin{quote}
  \color{violet}
  Lecture showed off the \texttt{which} command, but that may have flown by.

  This is a great example of something that should have been manageable to
  figure out on your own.
  The first google result for ``ls command not found'', is a stack exchange
  post where (admittedly the second response) says ``To check that it is
  indeed a problem with your path, what's the result of \texttt{/bin/ls} ?''
  (The query ``what does PATH do'' is equally fruitful).

  Another good way of problem-solving this would be to open a new terminal and
  look at the original contents (i.e.\ \texttt{echo \$PATH}). The
  colon-separated list is a little cryptic, but exploring those directories
  would hopefully let you figure this out.
\end{quote}

\textbf{Give an example of a command that works the same even with an empty
\texttt{PATH}. Why does this command still work?}
\begin{quote}
  \color{violet}
  Anything built-in to bash would still work. \texttt{cd} is the easy answer
  here (why doesn't it make sense for \texttt{cd} to be an extrenal command?).
  For a full list, you can type \texttt{help} and see a list of built-ins
  (incidentally, \texttt{help} itself is a built-in and would have been a
  good answer to this question).

  A interesting last question to add would have been, \emph{Why do you need to
    type \texttt{./a.out} to run your programs instead of simply
  \texttt{a.out}? What is the dot in \texttt{./} doing?}
\end{quote}


\section{Playing with the shell a bit: Special Variables}

Bash has quite a few special variables that can be very useful when writing
scripts or while working at the terminal.

\textbf{What does the variable \texttt{\$?} do? Give an example where this
value is useful}
\begin{quote}
  \color{violet}
  \texttt{\$?} holds the return value of the last command run. By convention,
  this is 0 if the command succeeded and nonzero if the command failed.
  It can be useful when writing scripts, especially if you want to have fairly
  complex things happen depending on success or failure (making it hard to
  just use \texttt{\&\&} or \texttt{||}).

  Try writing a simple program and varying the return value from
  \texttt{main}. Run your program and look at the value of \texttt{echo \$?}.
\end{quote}

\textbf{What does the variable \texttt{\$1} do? Give an example where this
value is useful}
\begin{quote}
  \color{violet}
  \texttt{\$1} is the first argument to a function. Bash syntax looks a little
  different that some other languages you may have seen before. You use can
  use it something like this:

  \begin{lstlisting}
$ cat b.sh
#!/usr/bin/env bash

# The first argument passed to this script
echo $1

# Argument 0 is the script itself
echo $0

function iamafunction {
       	echo "In iamafunction"
       	# Notice, no argument list, just assume they're there
       	echo "   Arg1 is ->$1<-"
       	echo "   Arg2 is ->$2<-"
       	echo "   Arg3 is ->$3<-"
}

# Can pass things through and in a different order, spaces separate arguments,
# no ()'s to call:
iamafunction one two three

# Can pass through variables too, note that $1 for the script is now $2 inside
# the called function and $3 is empty, so it just prints nothing
iamafunction firstarg $1

[-bash] Fri 07 Oct 19:39 [/tmp/bb]
$ ./b.sh arg1 arg2
arg1
./b.sh
In iamafunction
   Arg1 is ->one<-
   Arg2 is ->two<-
   Arg3 is ->three<-
In iamafunction
   Arg1 is ->firstarg<-
   Arg2 is ->arg1<-
   Arg3 is -><-
  \end{lstlisting}
\end{quote}


\newpage
\section{Basic Scripting}

Recall from lecture that scripting is really just programming, only in a very
high-level language. Interestingly, \texttt{sh} is probably one of the oldest
languages in regular use today.

\texttt{make} is a good tool for build systems, but we can actually use some
basic scripting to accomplish a lot of the same things.
First, write a simple C program that prints ``Hello World!''.
Write a shell script named \texttt{build.sh} that performs the following
actions:
\begin{enumerate}
  \item Compile your program
  \item Runs your program
  \item Verifies that your program outputs exactly the string ``Hello World!''
    \begin{itemize}
      \item There are good utilities that check the \texttt{\ul{diff}}erence of two
        files. They could be helpful.
    \end{itemize}
  \item Prints the string ``All tests passed.'' If the output is correct, or
    prints ``Test failed. Expected output $>>$Hello World$<<$, got
    output $>>$\{the program output\}$<<$''.
\end{enumerate}

\begin{quote}
  \color{red}
  I think this question would have been improved with a little more direction
  to help you get started, specifically, I would add:

  To get you started, here's a simple shell script. Try copying it and running it.
  \begin{lstlisting}
  #!/usr/bin/env bash

  echo "Hello, I am a shell script"

  # Anything after a # character is a comment

  # Shell scripts are like working in a terminal, only the script types the
  # commands automatically for you

  pwd
  ls

  # You won't break anything with shell scripts. The best way to write one is
  # to add some commands and run it until the script does what you want.
  \end{lstlisting}

  Recall that your shell script began life a simple text file, it's up to you
  to let the operating system know that this is actually now an executable
  program. ($\leftarrow$ this is a pointer to \texttt{chmod +x} without
  actually saying it)
\end{quote}

\st{\textbf{Copy the output of \texttt{cat build.sh} here:}}
\textbf{\color{red} Show that your script works as expected for both correct
and buggy hello world programs. Copy the contents of your script below:}


\begin{quote}
  \color{violet}

  As for a solution, there are \textbf{many} ways of answering this question.
  This is true of scripting in general. Scripts are (usually) not meant to be
  ``pretty'' programs. There is no need to have the shortest or most elegant
  solution. That's a waste of everyone's time. They are the kind of tool where
  you simply find something that works and move on.

  To that end, I've copied in several different solutions from several
  different students that show off the various ways this could be done on the
  next page.

  \newpage
  \lstset{tabsize=1}
  \lstset{basicstyle=\footnotesize\ttfamily}
  \begin{lstlisting}
  Probably my favorite submission, as the citation history at the bottom shows
  how this student identified the three challenges (command output -> variable,
  conditionals in bash, and string comparison in bash), found examples for each,
  and synthesized a result:

  #!/bin/bash
  #1. Compile the program
  gcc main.c -o hello_world

  #2. Run the program
  output_val=$(./hello_world)
  expected_val='Hello World!'

  #3. Verify the program outputs 'Hello World!'
  if [ "$output_val" == "$expected_val" ]
  then
    echo "All tests passed"
  else
    echo "Test failed. Expected output >>Hello World<<, got output >>$output_val<<"
  fi

  Sources:
  http://www.cyberciti.biz/faq/unix-linux-bsd-appleosx-bash-assign-variable-command-output/
  http://www.thegeekstuff.com/2010/06/bash-if-statement-examples/
  http://stackoverflow.com/questions/4277665/how-do-i-compare-two-string-variables-in-an-if-statement-in-bash

  --------------------------------------------------------------------------

  #!/bin/bash
  # Compile and make executable
  gcc hello.c -o hello_exec

  # Set the desired value in a temp variable
  test="Hello World!"

  # Store the output in a temp variable
  output="$(./hello_exec)"

  # Run diff against the desired string
  diff <(echo $output) <(echo $test) &>/dev/null

  # Check the exit code of diff
  if [ $? = 0 ]; then
    echo "All tests passed."
  else
    echo "Test failed. Expected output >>$test<<, got output>>$output<<"
  fi

  --------------------------------------------------------------------------

  $ cat build.sh
  g++ hello.cpp -o hello
  "Hello World!" > correct
  ./hello > output
  cmp --silent output correct\
    && echo "All tests passed"\
  || echo "Test Failed. Expected output >>Hello World!<<, got output >>$(cat output)<<"

  --------------------------------------------------------------------------

  gcc main.c -o main
  ./main > file1
  echo -n "Hello World!" > file2
  if diff -q file1 file2 > /dev/null; then
    echo "All tests passed."
  else
    echo -n "Test failed. Expected output >>Hello World!<<, got output >>"
    cat file1
    echo "<<"
  fi

  \end{lstlisting}
\end{quote}


\vfill

\section{Controlling your environment}

In lecture, we added a directory to our \texttt{PATH} so that we could just
type \texttt{hello} and the Hello World program would run. It would be
annoying to update the \texttt{PATH} variable every time we open a new
terminal. Fortuntately, we can do better.

\textbf{Describe how you would set up your system to modify your \texttt{PATH}
  automatically every time you open a new terminal (what file would you change
  and what would you put in it?)}
\begin{quote}
  \color{violet}
  Any of the files that are read on startup would work, but the traditional
  one to use is \texttt{.bashrc}

  You'll need to include \texttt{export} such any changes are propogated into
  subshells. Something like:

  \texttt{export PATH=\$PATH:~/bin}

  is quite common.
\end{quote}

\textbf{Roughly how long did you spend on this assignment? \rule{5cm}{0.1mm}}
\end{document}
