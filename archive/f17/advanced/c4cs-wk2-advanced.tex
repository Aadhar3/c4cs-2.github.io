\documentclass{article}
\usepackage[T1]{fontenc}

\usepackage{amssymb}
\usepackage{courier}
\usepackage{fancyhdr}
\usepackage{fancyvrb}
\usepackage[top=.75in, bottom=.75in, left=.75in,right=.75in]{geometry}
\usepackage{graphicx}
\usepackage{lastpage}
\usepackage{listings}
\lstset{basicstyle=\small\ttfamily}
\usepackage{mdframed}
\usepackage{parskip}
\usepackage{soul}
\usepackage{tabularx}
\usepackage{textcomp}
\usepackage{upquote}
\usepackage{xcolor}

% http://www.monperrus.net/martin/copy-pastable-ascii-characters-with-pdftex-pdflatex
\lstset{
upquote=true,
columns=fullflexible,
keepspaces=true,
literate={*}{{\char42}}1
         {-}{{\char45}}1
         {^}{{\char94}}1
}
\lstset{
  moredelim=**[is][\color{blue}\bf\small\ttfamily]{@}{@},
}

% http://tex.stackexchange.com/questions/40863/parskip-inserts-extra-space-after-floats-and-listings
\lstset{aboveskip=6pt plus 2pt minus 2pt, belowskip=-4pt plus 2pt minus 2pt}



\usepackage[colorlinks,urlcolor={blue}]{hyperref}
\usepackage[capitalise,nameinlink,noabbrev]{cleveref}
\crefname{section}{Question}{Questions}
\Crefname{section}{Question}{Questions}

\begin{document}


\fancyfoot[L]{\color{gray} C4CS -- F'17}
\fancyfoot[R]{\color{gray} Revision 1.0}
\fancyfoot[C]{\color{gray} \thepage~/~\pageref*{LastPage}}
\pagestyle{fancyplain}

\title{\textbf{Advanced Homework 2\\}}
\author{\textbf{\color{red}{Due: Wednesday, September 27th, 11:59PM (Hard Deadline)}}}
\date{}
\maketitle


\section*{Submission Instructions}
To receive credit for this assignment you will need to stop by someone's
office hours, demo your running code, and answer some questions. \textbf{\color{red}{Make sure
to check the office hour schedule as the real due date is at the last office
hours before the date listed above.}} This applies to assignments that need to be gone over with a TA only.
\textbf{Extra credit is given for early turn-ins of advanced exercises. These details can be found on the website under the advanced homework grading policy.}


\section{Git Golf}

Sometimes, when using git, people make mistakes. Git can be remarkably
forgiving, if you've ever told git to remember something, then you can usually
find a way to undo your mistake and get it back.

\begin{quote}
\begin{lstlisting}
# Grab a copy of the files for this question
> wget http://c4cs.github.io/static/f17/advanced/golf.tar.gz
\end{lstlisting}
\end{quote}

\subsection*{\#1: Undeleting Files}

In this repository, someone cleaning up ran
\texttt{git~rm~*.png}, which deleted every image, and committed the result.
Now the website is broken because an image that should have stayed was
deleted. While you could use \texttt{git~revert} to undo the commit, it also
changed the website source, so you really don't want to undo everything.

\noindent
\textbf{Demonstrate how to recover the missing picture \ul{without} reverting
the commit that deleted it.}


\subsection*{\#2: Undeleting Commits}

Sometimes it is possible to lose a commit. This can happen because you deleted
a branch, a rebase went poorly, or a reset went awry. Regardless of how it
happened, there are ways of finding commits that nothing is currently pointing
to. This repository has such a commit.

\noindent
\textbf{Demonstrate how to recover a commit that nothing is currently pointing to.}
% n.b. for the pedantic, the r****g is actually pointing to this commit, so
% it'll never be gc'd, but that's outside the scope of this question


\subsection*{\#3: Undeleting Changes}

When working with git, \texttt{git~add~my\_file} stages a file, but it isn't
actually committed until you run \texttt{git~commit}. Sometimes you change
your mind after a \texttt{git~add} and run \texttt{git~reset~my\_file} to
unstage a file. The changes to that file are still there, however. To really
undo changes, you use \texttt{git~reset~-{}-hard}.

\medskip
\noindent
Once you've been using these commands for a little while, it can be a little
too easy to accidentally type the wrong thing. In this repository, someone
accidentally typed \texttt{git~reset~-{}-hard} when they meant to just type
\texttt{git~reset}. Fortunately, because they had already run \texttt{git~add}
to stage their changes, the deleted changes can be recovered.

\noindent
\textbf{Demonstrate how to recover changes that had been staged for commit,
but were then deleted.}


\subsection*{Submission checkoff:}
\begin{itemize}
  \item[$\square$] Explain how you solved each problem
\end{itemize}


\newpage
\section{Automating Professionalism}

One feature of most version control systems is \emph{hooks}. A hook is an
automated script or tool that runs at various points in time. For example,
later in this course we will show how you can automatically run test cases
every time anyone commits code.

\medskip
\noindent
A slightly easier one to wire up, however, is a hook that will automatically
check the spelling of your commit messages for you, letting you know if you
made any mistakes. While spelling rarely counts for a class projects, it
adds a nice bit of professionalism to any future work you'll share with
others.

\medskip
\noindent
Like the regular homework, I recommend creating a temporary git repository to
play with while you test things and try to get things working.

\medskip
\noindent
Git stores all of its information and configuration in a folder named
\texttt{.git} in the root of each project. Navigate to \texttt{.git/hooks} and
rename \texttt{commit-msg.sample} to \texttt{commit-msg}. This activates the
commit message hook, which runs after you write and save your commit message.
Now go back to your repository, make a change, and commit it. Hmm, doesn't
look like anything changed. Make another change, but this time make this your
commit message (\textbf{exactly this, capitalization matters}):
\begin{quote}
  \begin{verbatim}
  This is a test.

  Signed-off-by: Me!
  Signed-off-by: Me!\end{verbatim}
\end{quote}
After trying to commit, type \texttt{git status}. What happened? Go back and
look at the commit hook. Do you understand what it does? Try running
\texttt{git commit --no-verify} with the same commit message. Go edit the
commit hook and delete the line \texttt{exit 1}. Try making a new commit with
the same commit message, what happens now? What is \texttt{\$1} in this
script?  (not sure? try adding lines like \texttt{echo~\$1} or
\texttt{echo~\$(file~\$1)} or \texttt{echo~\$(cat~\$1)} to the hook and making
commits, what happens?)

\medskip
\noindent
\texttt{aspell} is a simple command-line utility that checks spelling. Install
it and play with it a little.

\medskip
\noindent
\textbf{Write a commit hook that checks the spelling of a commit message.}
Your hook should not prevent the commit from going through (that'd be
annoying..), but it should print out any spelling errors. Your hook should
also print a message that suggests running the command
\texttt{git~commit~-{}-amend}\footnote{%
  Amending a commit lets you change your most recent commit. This is fine when
  you are working locally, but can be a dangerous command if you are sharing
  your repository with anyone else. Use with caution.
} if any errors are present.

\medskip

\subsection*{Some tips}
\begin{itemize}
  \item This is a little intimidating to get started. \emph{Try some things.}
    Make a bunch of garbage commits, modify the hook a little, see what
    happens.
  \item A shell script is exactly like working in a terminal. The only magic
    is that you've typed all the commands in advance instead of one at a time.
    So try some things in your terminal! Mess around until you get some
    commands that do what you want, then copy them to your script.
  \item Shell scripting is hard and ugly though. Maybe write a Python script
    to help you out and call it? Then again, calling commands from Python is
    kinda hard, so maybe not. \emph{Whatever you are more comfortable with.}
  \item There is \textbf{zero} need to be efficient. This hook is called
    rarely and operates on hundreds of bytes of text. Read the file 6 times.
    Write 7 temporary files. \emph{Who cares.} The goal is not to be pretty,
    the goal is to work.
  \item Speaking of temporary files, the \texttt{/tmp} directory can be a
    great place to throw those. There's even a command called \texttt{mktemp}
    that will give you a new, unique filename. (You can also do this without
    making any temporary files, it's just a little harder)
\end{itemize}

\subsection*{Submission checkoff:}
\begin{itemize}
  \item[$\square$] Show off your spellchecking hook in action, explain how it
    works
\end{itemize}

\subsubsection*{\emph{Optional} Extra Credit (1 pt)}
\begin{itemize}
  \item[$\square$] Demonstrate that your hook ignores errors in indented
    lines, explain how it works
\end{itemize}


\paragraph{Further exploration and some gotchas:}
\begin{itemize}
  \item Take a look at some of the other hooks, are any of them useful?
  \item For a list of all available hooks, type \texttt{man githooks}.
  \item Hooks have to be marked as executable to run (\texttt{chmod +x}). The
    sample hooks already had the executable bit set, which is why renaming the
    existing sample worked above.
  \item Hooks can call other scripts. Because invocation of hooks is
    controlled by the name of the script, if you want multiple scripts to run
    for a single hook, you'll need to have one script named correctly that
    calls your other hooks.
\end{itemize}

\end{document}
