\documentclass{article}
\usepackage[T1]{fontenc}

\usepackage{amssymb}
\usepackage{courier}
\usepackage{fancyhdr}
\usepackage{fancyvrb}
\usepackage[top=.75in, bottom=.75in, left=.75in,right=.75in]{geometry}
\usepackage{graphicx}
\usepackage{lastpage}
\usepackage{listings}
\lstset{basicstyle=\small\ttfamily}
\usepackage{mdframed}
\usepackage{parskip}
\usepackage{soul}
\usepackage{tabularx}
\usepackage{textcomp}
\usepackage{upquote}
\usepackage{xcolor}

% http://www.monperrus.net/martin/copy-pastable-ascii-characters-with-pdftex-pdflatex
\lstset{
upquote=true,
columns=fullflexible,
keepspaces=true,
literate={*}{{\char42}}1
         {-}{{\char45}}1
         {^}{{\char94}}1
}
\lstset{
  moredelim=**[is][\color{blue}\bf\small\ttfamily]{@}{@},
}

% http://tex.stackexchange.com/questions/40863/parskip-inserts-extra-space-after-floats-and-listings
\lstset{aboveskip=6pt plus 2pt minus 2pt, belowskip=-4pt plus 2pt minus 2pt}



\usepackage[colorlinks,urlcolor={blue}]{hyperref}
\usepackage[capitalise,nameinlink,noabbrev]{cleveref}
\crefname{section}{Question}{Questions}
\Crefname{section}{Question}{Questions}

\begin{document}


\fancyhead[C]{\hl{Select the right page in Gradescope or we will not grade the question!}}
\fancyhead[L]{}
\fancyhead[R]{}

\fancyfoot[L]{\color{gray} C4CS -- F'17}
\fancyfoot[R]{\color{gray} Revision 1.0}
\fancyfoot[C]{\color{gray} \thepage~/~\pageref*{LastPage}}
\pagestyle{fancyplain}

\title{\textbf{Homework 7\\Build Systems}}
\author{\textbf{\color{red}{Due: Wednesday, November 1st, 11:59PM (Hard Deadline)}}}
\date{}
\maketitle


\section*{Submission Instructions}
Submit this assignment on \href{https://gradescope.com/courses/9999}{Gradescope}.
You may find the free online tool \href{https://www.pdfescape.com}{PDFescape}
helpful to edit and fill out this PDF.
You may also print, handwrite, and scan this assignment.


\newpage
\begin{mdframed}\centering
For this assignment, we will experiment in the EECS\,280~W15 repository you
created for Homework~2.
\end{mdframed}

\section{Expressing Dependencies}
\begin{enumerate}\small
  \item Run \texttt{make} to build everything.
  \item Run \texttt{make} again (nothing happens).
  \item Edit \texttt{p2.cpp} and make a change (add a comment or something).
  \item Run \texttt{make} again.
  \item Edit \texttt{p2.h} and make a change (add a comment or something).
  \item Run \texttt{make} again.
\end{enumerate}

\textbf{Why did make rebuild things after step~4 but not after step~6? Why is
this a problem?}
\vspace{2cm}

\textbf{Rewrite the rule for \texttt{simple\_test} so that make rebuilds
correctly for any changes you make:\footnote{You do not need to worry about
  system header files}}
% (Original rule):
%\begin{lstlisting}
%simple_test: simple_test.cpp p2.cpp Recursive_list.cpp Binary_tree.cpp \
%	      recursive.cpp test_helpers.cpp
%	g++ -Wall -Werror -pedantic -O2 \
%	      simple_test.cpp p2.cpp Recursive_list.cpp Binary_tree.cpp \
%	      recursive.cpp test_helpers.cpp -o simple_test
%\end{lstlisting}
\vspace{2cm}

\emph{\small Think this is a pain? Check out the advanced homework for a
  BetterWay$^{TM}$.}


\section{From Build Engine to Rules Engine}

Makefiles are often asked to do more than simply build your software. A common
example is a rule named \texttt{clean} that deletes everything built by the
Makefile.

\begin{enumerate}\small
  \item Run \texttt{make} to build everything.
  \item Run \texttt{make clean} to delete everything that was built.
  \item Run \texttt{make} to build everything.
  \item Run \texttt{touch clean}
  \item Run \texttt{ls} {(\footnotesize do you understand what \texttt{touch} does?)}
  \item Run \texttt{make clean}
\end{enumerate}

Why did make run the \texttt{clean} rule after step~2 but not after step~5?

\textbf{What flag can you pass to make so that it unconditionally ``builds''
  the \texttt{clean} target?}

\texttt{make -\_\_ clean}

\textbf{Describe how to fix the Makefile so that fake targets like \texttt{clean} work correctly.}
\vspace{2cm}

\section{Removing Duplicated Effort}

Notice that currently the \texttt{all} target and the \texttt{test} target
have the same list of dependencies.

\textbf{List all the changes you have to make to the Makefile so that the
  \texttt{test} target correctly depends on the \texttt{all} target in all
  cases.}
\vspace{3cm}

\section{Anything Special about All?}
\label{q:all}

Currently, if you just type \texttt{make}, make will run \texttt{make all}.
One might wonder why make chooses the \texttt{all} goal by default.
While you could look this up, we are computer \emph{scientists}.
Make changes to the Makefile until you are confident that you understand how
make chooses the default goal.

\textbf{Describe the experiments you ran in order to determine what target
  make builds by default.}
\vspace{3cm}

\newpage

\section{Manipulating \texttt{make}'s environment}

One neat feature of \texttt{make} is that it ships with a large number of
\emph{implicit rules}. \texttt{make} understands that
\texttt{foo.c}$\rightarrow$\texttt{foo.o}$\rightarrow$\texttt{foo} without you
writing any rules. In fact, you can actually run \texttt{make} without a
\texttt{Makefile}! Let's play with this a little.


First let's get a simple environment set up and try some things out:
\begin{lstlisting}
@>@ mkdir /tmp/c4cs-build-systems && cd $_
@>@ echo -e '#include <stdio.h>\n\nint main() {\n\tprintf("Howdy\\n");\n\treturn 0;\n}\n' > hello.c
@>@ cat hello.c   # just so you can see what that did
@>@ make
@>@ make hello
@>@ ./hello
\end{lstlisting}
Does \texttt{make clean} work? Why not?

Now try
\begin{lstlisting}
@>@ rm hello
@>@ make -r hello
\end{lstlisting}
What does the \texttt{-r} flag do?

Next try
\begin{lstlisting}
@>@ rm hello
@>@ make CFLAGS=-03 hello
\end{lstlisting}
What changed when hello was built this time?

Finally run
\begin{lstlisting}
@>@ make hello -p | less
\end{lstlisting}

\medskip
\textbf{Make an educated guess at which built-in rule is used to create
  ``hello'' from ``hello.c'' and copy it here. What makes you think this rule
  is responsible?}
%(can you also find where the default goal set from \cref{q:all}?)
\vspace{2.5cm}


Now let's add an additional file to the mix, only a C++ file this time:
\\{(\footnotesize This example uses a
  \href{https://en.wikipedia.org/wiki/Here_document\#Unix_shells}{special
    shell syntax} for easily writing multiple lines to a shell command)}
\begin{lstlisting}
@>@ cat << MARKER > wazzup.cpp
#include <iostream>

int main() {
  std::cout << "Wazzup?" << std::endl;
  return 0;
}
MARKER
@>@ cat wazzup.cpp
\end{lstlisting}

\textbf{Using what you have learned, write a single make command (i.e. only
  call make once) that, without a Makefile, will build both ``hello'' and
  ``wazzup'', but builds hello optimized for speed (\texttt{-O3}) and wazzup
  optimized for size (\texttt{-Os}).
  \emph{\small Hint: One is a \texttt{C} program and one is a \texttt{C++} program\dots}
}



\vfill

\noindent
\textbf{Roughly how long did you spend on this assignment? \rule{5cm}{0.1mm}}
\medskip

\end{document}
