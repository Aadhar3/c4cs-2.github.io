\documentclass{article}
\usepackage[T1]{fontenc}

\usepackage{amssymb}
\usepackage{courier}
\usepackage{fancyhdr}
\usepackage{fancyvrb}
\usepackage[top=.75in, bottom=.75in, left=.75in,right=.75in]{geometry}
\usepackage{graphicx}
\usepackage{lastpage}
\usepackage{listings}
\lstset{basicstyle=\small\ttfamily}
\usepackage{mdframed}
\usepackage{parskip}
\usepackage{soul}
\usepackage{tabularx}
\usepackage{textcomp}
\usepackage{upquote}
\usepackage{xcolor}

% http://www.monperrus.net/martin/copy-pastable-ascii-characters-with-pdftex-pdflatex
\lstset{
upquote=true,
columns=fullflexible,
keepspaces=true,
literate={*}{{\char42}}1
         {-}{{\char45}}1
         {^}{{\char94}}1
}
\lstset{
  moredelim=**[is][\color{blue}\bf\small\ttfamily]{@}{@},
}

% http://tex.stackexchange.com/questions/40863/parskip-inserts-extra-space-after-floats-and-listings
\lstset{aboveskip=6pt plus 2pt minus 2pt, belowskip=-4pt plus 2pt minus 2pt}



\usepackage[colorlinks,urlcolor={blue}]{hyperref}
\usepackage[capitalise,nameinlink,noabbrev]{cleveref}
\crefname{section}{Question}{Questions}
\Crefname{section}{Question}{Questions}

\begin{document}


\fancyfoot[L]{\color{gray} C4CS -- F'17}
\fancyfoot[R]{\color{gray} Revision 1.0}
\fancyfoot[C]{\color{gray} \thepage~/~\pageref*{LastPage}}
\pagestyle{fancyplain}

\title{\textbf{Homework 11\\IDEs}}
\author{\textbf{\color{red}{Due: Wednesday, November 29th, 11:59PM (Hard Deadline)}}}
\date{}
\maketitle


\section*{Submission Instructions}
Submit screenshots with 1-3 descriptive paragraphs of what is shown, and how it
was a learning experience for you. Alternately, visit a TA during office hours
and have a show and tell about what you did and learned.


\section{Which IDE? Which Platform?}

The proprietary nature of the two most common IDEs (Xcode and Visual Studio), and their mutually exclusive availability will make this assignment be a
``Choose Your Own Adventure.'' Combine that with the fact that neither is
available on open source operating systems, and the work you will do for this
assignment quite subjective. The goal however is straightforward, sample a new
IDE, or extend your abilities with your current IDE.

\subsection*{The Assignment}
In lecture, we discussed the many tools and services that modern IDEs provide.
This list included far more than the ``Big Five,'' which are text editor,
compiler, debugger, file manager, and build system. Your task is to show how
you can use more of your IDE. To do this, you will need to demonstrate the use
of 2 or more of the other tools or functionalities that your IDE provides.

\subsection{I'm a Mac}
Some examples of new uses:
\begin{enumerate}
\item Use ``Source Control'' to put your project and source code into a local
      and an online repository (gitlab.eecs.umich.edu).
\item Start a new type of project that is more than just command line program
      (preference pane, game, automator action, an iOS/watchOS/tvOS application
      or game). This product can be simple, make sure that it launches and
      runs, and displays your name, uniqname, and ``C4CS''.
\item Add more target(s) to an existing Xcode project, that can be used to very
      thoroughly test aspects of the main target.
\item Toy with something in Objective-C or Swift.
\end{enumerate}

\subsection{I'm a PC}
Some examples of new uses:
\begin{enumerate}
\item Install ``Git Client Tools'' and use VS to put your project and source
      code into a local and an online repository (gitlab.eecs.umich.edu).
\item Start a new type of project that is more than just command line program
      (web, Silverlight or Office tool, a Windows application or game). This
      product can be simple, make sure that it launches and runs, and displays
      your name, uniqname, and ``C4CS''.
\item Add more projects(s) to an existing Visual Studio Solution, that can be
      used to very thoroughly test aspects of the main target.
\item Try something in a new language.
\end{enumerate}

\subsection{I am a Full Time Linux User}
Some examples of new uses:
\begin{enumerate}
\item Install an IDE (NetBeans, Eclipse, etc). You'll need to ``BYOC,'' bring
      your own compiler.
\item Use your local IDE to put your project and source code into a local and
      an online repository (gitlab.eecs.umich.edu).
\item Start a new type of project that is more than just command line program
      (desktop app, control panel or utility, etc). This product can be simple,
      make sure that it launches and runs, and displays your name, uniqname,
      and ``C4CS''.
\item Add more projects(s) to an existing IDE Solution, that can be used to
      very thoroughly test aspects of the main target.
\item Try something in a new language.
\end{enumerate}

\subsection{That Still Doesn't Cover My Situation}
Get creative! Try something, learn something, and make sure that you can demonstrate this to a TA:
\begin{enumerate}
\item Install an IDE into your virtual machine's OS.
\item Install some other IDE into your host OS.
\item Keep trying...
\end{enumerate}

\subsection*{Be Honest}
This is going to be a very difficult assignment to grade since everyone has
different hardware, experiences, and workflows. It would be very easy to say
that you've never used an IDE, so just installing one is new to you. Push
yourself to learn a new and useful skill, that will help you moving forward as
a Computer Scientist.

\end{document}
