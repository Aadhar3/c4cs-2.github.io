\documentclass{article}
\usepackage{amssymb}
\usepackage{courier}
\usepackage{fancyhdr}
\usepackage{fancyvrb}
\usepackage[T1]{fontenc}
\usepackage[top=.75in, bottom=.75in, left=.75in,right=.75in]{geometry}
\usepackage{graphicx}
\usepackage{lastpage}
\usepackage{listings}
\lstset{basicstyle=\small\ttfamily}
\usepackage{soul}
\usepackage{upquote}
\usepackage{xcolor}

\usepackage[colorlinks,urlcolor={blue}]{hyperref}

\begin{document}

\fancyfoot[L]{\color{gray} C4CS -- W'16}
\fancyfoot[R]{\color{gray} Revision 1.0}
\fancyfoot[C]{\color{gray} \thepage~/~\pageref*{LastPage}}
\pagestyle{fancyplain}


\title{\textbf{Advanced Homework 11\\}}
\author{Assigned: Friday, March 25, 11:00AM}
\date{\textbf{\color{red}{Due: Before the first lecture on Friday, April 8}}}
\maketitle


\section*{Submission Instructions}
To receive credit for this assignment you will need to stop by someone's
office hours, demo your running code, and answer some questions.

\section{Topic Discovery}
Write a script that takes a search term, an integer, and any number of files on
the command line, and prints out which files have the search term at least that
many times.

For example:

\begin{lstlisting}
$ cat file1.txt
I love algorithms! Analyzing algorithmic complexity is my favorite field of CS.

$ cat file2.txt
This is a file with only one use of the word algorithm.

$ cat file3.txt
I hate algorithms!
I dislike algorithms so much that I have written this treatise against algorithms.

$ python topic.py algorithm 2 file1.txt file2.txt file3.txt
file1.txt
file3.txt
\end{lstlisting}


\newpage
\section{Iterator Protocol}
As a very dynamic language, Python has a concept called ``protocols''. The
idea here is that language constructs, such as a for loop, are really just
calling some well-defined functions under the hood.

\medskip
\noindent
Iterable objects in Python are things that make sense to loop over in a for
loop, for example strings are iterable:
\begin{lstlisting}
  >>> for letter in "abc":
  ...     print(letter)
  ...
  a
  b
  c
\end{lstlisting}
As are tuples:
\begin{lstlisting}
>>> for number in (1, 2, 3):
...     print (number)
...
1
2
3
\end{lstlisting}
Many functions, such as \texttt{sum} also accept iterable objects:
\begin{lstlisting}
>>> sum((1,2,3))
6
\end{lstlisting}

\medskip
\noindent
Any object can be iterated if it implements a \texttt{\_\_iter\_\_} function
(similar to \texttt{begin()} on a C++ STL container) and a \texttt{\_\_next\_\_}
function (similar to the \texttt{++} operator on a C++ STL iterator).

\medskip
\noindent
\url{http://stackoverflow.com/questions/9884132/what-exactly-are-pythons-iterator-iterable-and-iteration-protocols}
gives another summary of Python iterators and pointers to some more resources
you may find helpful.


\vspace{3cm}
\noindent
Modify your script from Question~1 such that the following works:
\begin{lstlisting}
>>> import glob   # This is a built-in python package
>>> import topic
>>> searcher = topic.Searcher()
>>> for filename in searcher.search('algorithm', 2, glob.glob('*.txt')):
...     print(filename)
...
file1.txt
file3.txt
\end{lstlisting}

\newpage
\section{Git Pre-Push Hook}
Suppose we're writing a program \lstinline{prog}, and have constructed a suite of regression tests that are named \lstinline{test*.cpp}, where \lstinline{*} is the wildstar (any string may be in its place). Write a script that:
\begin{enumerate}
    \item Compiles the main program, you may compile it how you like. Stop if there are compilation errors.
    \item Compiles all the test cases, again compile as you like and stop if there are compilation errors. 
    \item Run the test case and provide feedback on how many succeeded and failed. 
    \item If any test case fails, stop the push. 
\end{enumerate}
\vspace{5cm}


\end{document}
