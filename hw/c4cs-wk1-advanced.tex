\documentclass{article}
%\usepackage{fullpage}
%\usepackage[top=1in, bottom=1in, left=1cm,right=1cm]{geometry}
\usepackage[top=.75in, bottom=.75in, left=.75in,right=.75in]{geometry}
\usepackage{fancyhdr}
\usepackage{color}
\usepackage[colorlinks,urlcolor={blue}]{hyperref}

\begin{document}

\fancyfoot[L]{C4CS -- W'16}
\fancyfoot[R]{Revision 1.0}
\pagestyle{fancyplain}


\title{\textbf{Advanced Homework 1\\A Different Kind of Virtual Machine}}
\author{Assigned: Friday, January 8, 11:00AM}
\date{\textbf{\color{red}{Due: Friday, January 15, 11:00AM (Hard Deadline)}}}
\maketitle

\noindent
\emph{Not all of the Advanced Homeworks are the same difficulty. I would rate
  this as one of the hardest ones (but cool!) in no small part because you
  have not been taught a lot of what you need to know to do this assignemt.
  I have asked the course staff not to help out with this assignment, but I
  am curious to see if there are some folks who are capable of doing it.
}

\section*{Submission Instructions}
To receive credit for this assignment you will need to stop by someone's
office hours and demo your running code.
You will need to explain all of the differences in the output of running
\texttt{file} on the native code versus the ARM code to the instructor.


\section{Emulating an ARM}

Using VirtualBox, we were able to simulate a whole computer, running a
computer as a program inside a running computer. Part of what helped us out
was that the simulated computer we were running was very similar to the
actual, underlying hardware. Critically, the guest operating system was built
for the same \href{https://en.wikipedia.org/wiki/Instruction_set}{ISA} as the
host machine. What if that's not the case, however? What if we wanted to
create a virtual machine for something like a smartphone, that uses an ARM
architecture, and run that simulation on our x64 machine?

It turns out we can do this! In fact, the Android development environment ships
with \href{http://developer.android.com/tools/devices/emulator.html}{a tool}
that does exactly that. For this assignment we aren't going to simulate a
whole phone, but we will use the same underlying tool,
\href{http://qemu.org}{QEMU}. The Quick Emulator (QEMU) is capable of
on-the-fly recompilation, or
\href{https://en.wikipedia.org/wiki/Binary_translation}{binary translation},
which allows it to run a program compiled for an ARM machine on an x64 machine
by rewriting ARM instructions as x64 instructions as they're executed. Today,
we're going to build a simple Hello World program for an ARM but run it on an
x64 machine.

\begin{enumerate}
  \item Write a simple Hello World program and save it as \texttt{main.c}
  \item Look at the output of \texttt{cc main.c -o main\_x64 \&\& file main\_x64}.
    Copy this down somewhere and save it for later for comparison.
  \item Install an ARM \href{https://en.wikipedia.org/wiki/Cross_compiler}{cross compiler}
  \item Compile \texttt{main.c} using the cross compiler. Run \texttt{file} on
    the resulting binary. How does it compare to the version that used the
    regular compiler?
  \item Install qemu-arm in your virtual machine (yes, a VM can run a VM,
    though binary translation is technically not creating a virtual machine)
  \item Run \texttt{qemu-arm main\_arm} to print ``Hello World'' by running
    translated ARM code as a native x64 process in a virtual machine!
\end{enumerate}

\end{document}
