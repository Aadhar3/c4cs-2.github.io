\documentclass{article}
\usepackage[T1]{fontenc}

\usepackage{amssymb}
\usepackage{courier}
\usepackage{fancyhdr}
\usepackage{fancyvrb}
\usepackage[top=.75in, bottom=.75in, left=.75in,right=.75in]{geometry}
\usepackage{graphicx}
\usepackage{lastpage}
\usepackage{listings}
\lstset{basicstyle=\small\ttfamily}
\usepackage{mdframed}
\usepackage{parskip}
\usepackage{soul}
\usepackage{tabularx}
\usepackage{textcomp}
\usepackage{upquote}
\usepackage{xcolor}

% http://www.monperrus.net/martin/copy-pastable-ascii-characters-with-pdftex-pdflatex
\lstset{
upquote=true,
columns=fullflexible,
keepspaces=true,
literate={*}{{\char42}}1
         {-}{{\char45}}1
         {^}{{\char94}}1
}
\lstset{
  moredelim=**[is][\color{blue}\bf\small\ttfamily]{@}{@},
}

% http://tex.stackexchange.com/questions/40863/parskip-inserts-extra-space-after-floats-and-listings
\lstset{aboveskip=6pt plus 2pt minus 2pt, belowskip=-4pt plus 2pt minus 2pt}



\usepackage[colorlinks,urlcolor={blue}]{hyperref}
\usepackage[capitalise,nameinlink,noabbrev]{cleveref}
\crefname{section}{Question}{Questions}
\Crefname{section}{Question}{Questions}

\begin{document}

\fancyhead[C]{\hl{Select the right page in Gradescope or we will not grade the question!}}
\fancyhead[L]{}
\fancyhead[R]{}

\fancyfoot[L]{\color{gray} C4CS -- W'16}
\fancyfoot[R]{\color{gray} Revision 1.1}
\fancyfoot[C]{\color{gray} \thepage~/~\pageref*{LastPage}}
\pagestyle{fancyplain}



\title{\textbf{Homework 10\\Profiling}}
\author{Assigned: Friday, March 18}
\date{\textbf{\color{red}{Due: Friday, March 25, 11:00AM (Hard Deadline)}}}
\maketitle


\section*{Submission Instructions}
Submit this assignment on \href{https://gradescope.com/courses/2248}{Gradescope}.
You must submit every page of this PDF.
We recommend using the free online tool \href{https://www.pdfescape.com}{PDFescape}
to edit and fill out this PDF.
You may also print, handwrite, and scan this assignment.

There may multiple answers for each question. If you are unsure,
state your assumptions and your reasoning for why you think your answer
makes sense.

\textbf{\hl{You must select the right page for each question on Gradescope.\\
    If you do not select the right pages, we will not grade the question.}}

\section*{Optional Reading}

\emph{Developer Survey 2016}, from Stack Overflow.

\url{https://stackoverflow.com/research/developer-survey-2016}

Some students have asked for some insights into CS in industry. Every year,
Stack Overflow does a large survey of developers. Their intro does a nice job
of covering its caveats, but it remains an interesting look into what it means
to have a career in CS and what CS in the real world looks like.


\section*{CAVEATS}

\texttt{gprof} is \textbf{not} a debugging tool. Your code \textbf{must} work
correctly before trying to run gprof. Your code \textbf{must} exit normally
(no segfault, exception, etc) or gprof will not do anything!

\texttt{gprof} \textbf{does not work} on macs. On mac, the best tool to use is
\href{https://developer.apple.com/library/watchos/documentation/DeveloperTools/Conceptual/InstrumentsUserGuide/index.html}{Instruments},
however, for this homework please use your VM and gprof.


\newpage
\section{Reasoning about performance}

Clone a copy of
\url{https://gitlab.eecs.umich.edu/ppannuto/c4cs-w16-profiling}.

Look through \texttt{main.c} to understand what this program is doing (not
much, a ``nop'' is ``no-operation'', that is, sit idle for a cycle). The call
graph of this program is about:
\begin{verbatim}
\-- main
    \-- parent
            \-- child1
            \-- child2
    \-- no_children
\end{verbatim}

The \texttt{time} utility reports how long a program takes to run.

Run \texttt{make~baseline}.

\textbf{How long did it take for the baseline to run?\\
  Given how long baseline took, how long do you expect each of the following
  to take from the start of each function until it finishes?}

\bigskip
\textbf{Baseline} \underline{\hspace{3cm}} seconds

\bigskip
\textbf{child1} \underline{\hspace{3cm}} seconds

\bigskip
\textbf{child2} \underline{\hspace{3cm}} nanoseconds $\leftarrow$ note the units here

\bigskip
\textbf{parent} \underline{\hspace{3cm}} seconds $\leftarrow$ think carefully about this one! What's weird about the for loops?

\bigskip
\textbf{no\_children} \underline{\hspace{3cm}} seconds



\section{\texttt{gprof} overhead}

Now run \texttt{make~overhead}.

\textbf{What is the difference between \texttt{main\_no\_profiling} and
  \texttt{main\_with\_profiling}?}
\vspace{2cm}

\textbf{Which ran faster, \texttt{main\_no\_profiling} or
  \texttt{main\_with\_profiling}? \ul{Why?}}
\vspace{3cm}

\textbf{Between child1 or child2, which should you remove such that
  \texttt{main\_no\_profiling} and \texttt{main\_with\_profiling} will have
  nearly the same runtime? \ul{Why} remove that child?}

\newpage
\section{Understanding \texttt{gprof}}
\label{understanding}

Finally, run \texttt{make~profile.txt}. Take a look at the
\texttt{profile.txt} file.

\textbf{The gprof output has many different times it reports, such as
  ``cumulative seconds'' and ``self seconds''. Which timing report matches
  what Question~1 asked for?}
\vspace{3cm}

\textbf{How well did your estimates from Question 1 line up with the time
  reported by gprof?\\
  What explains the difference between your estimate and
  the time reported by gprof?}
\vspace{4cm}

\textbf{What generated the file \texttt{gmon.out}?}
\vspace{3cm}


\newpage
\section*{Document Revision History}
March 21, 5PM: Revise the first question in section \ref{understanding}.

\textbf{\st{Did Question 1 ask for ``cumulative seconds'' or ``self seconds''?  What is the difference?}} 
$\rightarrow$
\textbf{The gprof output has many different times it reports, such as
  ``cumulative seconds'' and ``self seconds''. Which timing report matches
  what Question~1 asked for?}

\end{document}
