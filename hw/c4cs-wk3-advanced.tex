\documentclass{article}
\usepackage{amssymb}
\usepackage{courier}
\usepackage{fancyhdr}
\usepackage[top=.75in, bottom=.75in, left=.75in,right=.75in]{geometry}
\usepackage{lastpage}
\usepackage{listings}
\lstset{basicstyle=\small\ttfamily}
\usepackage{xcolor}

\usepackage[colorlinks,urlcolor={blue}]{hyperref}

\begin{document}

\fancyfoot[L]{\color{gray} C4CS -- W'16}
\fancyfoot[R]{\color{gray} Revision 1.0}
\fancyfoot[C]{\color{gray} \thepage~/~\pageref*{LastPage}}
\pagestyle{fancyplain}


\title{\textbf{Advanced Homework 3\\Bridging the Editor/IDE Gap}}
\author{Assigned: Friday, January 22, 11:00AM}
%\date{\textbf{\color{red}{Due: Before the end of the final office hours this week (Thursday, $\sim$8:15PM)}}}
\date{\textbf{\color{red}{Due: Before the end of the final office hours this week (Thursday, $\sim$5:30PM)}}}
\maketitle


\section*{Submission Instructions}
To receive credit for this assignment you will need to stop by someone's
office hours, demo your running code, and answer some questions.

\medskip
\noindent
For this assignment, you may use vim or Emacs, whichever you are more
comfortable with.

\section{Some Tools that Support Other Tools}

Integrated Development Environments (IDEs), such as Visual Studio, Xcode, or
Eclipse provide many features beyond simple text editing.\footnote{
  Interestingly, ``simple text editing'' is often what IDEs are worst at
  out-of-the-box. Hence part~3 of this assignment.
}
Some of the most useful features rely on an understanding of C/C++ code,
allowing you to jump from a function's use to its definition, inspect the types
of complex structures, and assist authoring by suggesting possible
completions.

Vim and Emacs both support rudimentary autocompletion out-of-the-box. Try
editing an existing file and start typing a function you've used before, then
hit \texttt{ctrl-n} and \texttt{ctrl-p} in vim or \texttt{M-/} in Emacs. This
autocompletion has no understanding of C, it simply does string matching with
any other strings anywhere else in the open files.\footnote{
  Vim also understands how to parse C \texttt{\#include} directives and Python
  \texttt{import} statements, and will search those files as well.
}

There are two utilities, ctags/etags (vim/Emacs respectively) and cscope that
can be integrated with your editor to provide some contextually aware editing
assistance.
ctags/etags works pretty much out of the box. Somewhat annoyingly, cscope
requires you to generate a list of files for it to process. You may find that
the \texttt{find} utility, and its filetype filters in particular, are very
helpful for generating this file list.

Find a project of reasonable complexity,\footnote{
  If you don't have a good project, sqlite
  \href{https://www.sqlite.org/2016/sqlite-src-3100200.zip}{(zip of source)}
  is generally regarded as a well-engineered example of code. However, I
  recommend trying on your own code first.
} set up ctags/etags and cscope, and integrate them with your editor of
choice.

\subsection*{Submission checkoff:}
\begin{itemize}
  \item[$\square$] Show off ctags usage
    \begin{itemize}
      \item[$\square$] What did you have to do to generate tags for ctags/etags?
      \item[$\square$] How often do you need to generate tags (after what
        changes do you need to regenerate tags)?
      \item[$\square$] What did you have to do to integrate ctags/etags with
        your editor?
    \end{itemize}
  \item[$\square$] Show off cscope usage
    \begin{itemize}
      \item[$\square$] How did you generate a file list for cscope?
      \item[$\square$] How often do you need to run cscope?
      \item[$\square$] What did you have to do to integrate cscope with your
        editor?
    \end{itemize}
  \item[$\square$] Explain the difference between ctags and cscope
    \begin{itemize}
      \item[$\square$] Is one strictly better than the other or is it useful
        in some cases to set up both? Why?
    \end{itemize}
\end{itemize}



\section{Extending Editors}

ctags/etags and cscope are programs that run independently of your text
editor. You can also extend your editor with plugins. For vim, plugins are
written in the custom language vimscript; Emacs itself is a Lisp interpreter,
and plugins are written in Emacs Lisp.

While writing a plugin is a great way to learn a lot about the internals of
your editor, it can be a steep learning curve. Fortunately, the Internet has
an enormous library of really interesting and powerful plugins.
As an example, there are plugins that solve the annoyance from part~1, that
will automatically regenerate tags whenever needed.

Find at least three plugins to install in your editor. You may find it is
easier to first install a plugin manager and then install plugins.

\subsection*{Submission checkoff:}
\begin{itemize}
  \item[$\square$] For each plugin:
    \begin{itemize}
      \item[$\square$] What does this plugin do?
      \item[$\square$] Show off how you use it
      \item[$\square$] Why did you install this plugin?
    \end{itemize}
\end{itemize}


\section{Everything Old is New Again}

While advanced IDEs and new text editors come out remarkably often, rarely are
their simple text editing and navigation capabilities as good as the
``classic'' editors this homework has focused on. For this reason, there are
plugins (or in some cases its built-in) to provide vim-like or Emacs-like
editing modes for nearly every other editor.

Pick your favorite ``advanced'' GUI editor (suggestions Visual Studio, Xcode,
Eclipse, Sublime, or Atom -- gvim doesn't count) and set it up to edit text in
vim or Emacs mode. You will likely need to install a plugin to do this.

\subsection*{Submission checkoff:}
\begin{itemize}
  \item[$\square$] Show off your plugin in action
\end{itemize}

\end{document}
