\documentclass{article}
\usepackage[T1]{fontenc}

\usepackage{amssymb}
\usepackage{courier}
\usepackage{fancyhdr}
\usepackage{fancyvrb}
\usepackage[top=.75in, bottom=.75in, left=.75in,right=.75in]{geometry}
\usepackage{graphicx}
\usepackage{lastpage}
\usepackage{listings}
\lstset{basicstyle=\small\ttfamily}
\usepackage{mdframed}
\usepackage{parskip}
\usepackage{soul}
\usepackage{tabularx}
\usepackage{textcomp}
\usepackage{upquote}
\usepackage{xcolor}

% http://www.monperrus.net/martin/copy-pastable-ascii-characters-with-pdftex-pdflatex
\lstset{
upquote=true,
columns=fullflexible,
keepspaces=true,
literate={*}{{\char42}}1
         {-}{{\char45}}1
         {^}{{\char94}}1
}
\lstset{
  moredelim=**[is][\color{blue}\bf\small\ttfamily]{@}{@},
}

% http://tex.stackexchange.com/questions/40863/parskip-inserts-extra-space-after-floats-and-listings
\lstset{aboveskip=6pt plus 2pt minus 2pt, belowskip=-4pt plus 2pt minus 2pt}

\usepackage[colorlinks,urlcolor={blue}]{hyperref}
\usepackage[capitalise,nameinlink,noabbrev]{cleveref}
\crefname{section}{Question}{Questions}
\Crefname{section}{Question}{Questions}

\begin{document}

\fancyhead[C]{\hl{Select the right page in Gradescope or we will not grade the question!}}
\fancyhead[L]{}
\fancyhead[R]{}

\fancyfoot[L]{\color{gray} C4CS -- W'17}
\fancyfoot[R]{\color{gray} Revision 1.0}
\fancyfoot[C]{\color{gray} \thepage~/~\pageref*{LastPage}}
\pagestyle{fancyplain}


\title{\textbf{Homework 6\\Practical Regexes and Synthesis of Rich Commands}}
\author{\textbf{\color{red}{Due: Saturday, February 18, 10:00PM (Hard Deadline)}}}
\date{}
\maketitle


\section*{Submission Instructions}
Submit this assignment on \href{https://gradescope.com/courses/3499}{Gradescope}.
You may find the free online tool \href{https://www.pdfescape.com}{PDFescape}
helpful to edit and fill out this PDF.
You may also print, handwrite, and scan this assignment.


\section{Playing with words}

\begin{minipage}{0.6\textwidth}
By default, Ubuntu ships with a few dictionaries. We can find them in the
\texttt{/usr/share/dict} directory. If we head into that directory, we can see
(\texttt{wc~-l~*}) that these are not small lists. American English comes in
just shy of 100,000 words.

\medskip
\noindent
An interesting file is \texttt{cracklib-small}. This is a word list of around
50,000 common passwords. We can use the \texttt{grep} utility to search
through this file quickly to see if our password is in the file. For example
if my password is ``password'', then \texttt{grep~password~cracklib-small} tells
me that I've picked a bad one, but \texttt{grep~sup3rs3cure~cracklib-small}
tells me that may be a better choice.

\medskip
\noindent
Remember \texttt{ed}? \href{https://en.wikipedia.org/wiki/Grep}{Wikipedia
tells us} that \texttt{grep}'s name, ``comes from the ed command g/re/p
({\bf g}lobally search a {\bf re}gular expression and {\bf p}rint)''. Let's
give that a go eh?  Run \texttt{ed~cracklib-small} and try the command
\texttt{g/password/p}. Look familiar?

\medskip
\noindent
Thus far, with \texttt{ed} in lecture and \texttt{sed} in the previous
homework, we've only used simple regular expressions, namely things that match
the whole string we're searching for. Regular expressions can be far more
powerful, however.

\medskip
\noindent
Try the following commands:
\begin{lstlisting}
> grep password cracklib-small
> grep pass cracklib-small
> grep ^pass cracklib-small   # That's a caret, shift+6
> grep pass$ cracklib-small
> grep ^pass$ cracklib-small
> grep pass^ cracklib-small
> grep '$pass' cracklib-small # Why do we need quotes here?
> grep p.ss cracklib-small
> grep ^..th$ cracklib-small
> # Play with some others of your own design
\end{lstlisting}


\end{minipage}
\begin{minipage}{0.05\textwidth}
  ~~~~
\end{minipage}
\begin{minipage}{0.35\textwidth}
\framebox{\parbox{\dimexpr\linewidth-2\fboxsep-2\fboxrule}{\footnotesize
  \textbf{Aside:}\\
  Programs that want to check your spelling will use the file
  \texttt{/usr/share/dict/words}. Notice, however, that when we type
  \texttt{ls} in this directory, the \texttt{words} file shows up in teal,
  indicating that it's a symlink. Recall that a symbolic link is a way to make
  something that looks like a real file, but is actually just a pointer to
  another file.

  \medskip
  We can follow this pointer, however, to see what the actual file is. Type
  \texttt{ls~-l~words} to see what it actually points to, in this case
  \texttt{/etc/dictionaries-common/words}. It turns out that this too is a
  symlink however! If we then type
  \texttt{ls~-l~/etc/dictionaries-common/words} we see that it points back to
  \texttt{/usr/share/dict/american-english}, which is finally a real file.

  \medskip
  These little circles come up sometimes for compatibility reasons. Some
  programs expect to find the word list at \texttt{/usr/share/dict/words}
  while other programs  expect it at \texttt{/etc/dictionaries-common/words}.
  Using symlinks we can make all of these point to the same file. We can also
  easily change the language used for spellchecking by all programs simply by
  changing what the last symlink points to.

  \medskip
  To shortcut this whole operation, we can use the \texttt{readlink} utility.
  Try the command \texttt{readlink~words}. Now try \texttt{readlink~-f~words}.
  Does what each of these commands are doing make sense?

  \medskip
  Simple primitives build powerful features.
}}
\end{minipage}

\newpage

\noindent
\textbf{What does a carat (\^{}) mean in a regular expression?}
\vspace{4em}

\noindent
\textbf{What does a dollar sign (\$) mean in a regular expression?}
\vspace{4em}

\noindent
\textbf{What does a period (.) mean in a regular expression?}
\vspace{4em}

\noindent
Sometimes it's more interesting to know how many matches there are, rather
than the exact matches themselves. \texttt{grep} provides the \texttt{-c}
(count) flag for this case. For example
\texttt{grep~-c~password~cracklib-small} tells us there are 3~lines that have
``password'' in them, but there are (\texttt{grep~-c~pass~cracklib-small})
60~lines with ``pass''.

\medskip
\noindent
\textbf{There are 810 lines in cracklib-small that are exactly 3 characters
long. Give a command you could run to learn that number:}
\vspace{4em}


\noindent
A ``count'' flag is a very common flag and very useful flag shared by many
utilities, not just grep.

\bigskip
\bigskip
\noindent
Groups are another powerful feature. Try
\begin{lstlisting}
> grep ^p[ao]ss cracklib-small
> grep ^p[aeo]ss cracklib-small
> grep ^p[a-z]ss cracklib-small
\end{lstlisting}

\medskip
\noindent
\textbf{There are 766 lines in cracklib-small that are exactly 3
\emph{letters} long. Give a command you could run to learn that number:}
\vspace{4em}


\medskip
\noindent
Command-line tools really start to shine when you string them together.
Try running the following command:
\begin{lstlisting}
> for vowel in a e i o u; do echo -e "$(grep -c ^$vowel cracklib-small) \t $vowel"; done | sort -rn
\end{lstlisting}
Try playing around with this command a bit. Remove the flags to \texttt{sort},
remove \texttt{sort} altogether, replace the body of the for loop (everything
between \texttt{do} and the \texttt{;}) with just \texttt{echo \$vowel}.

\smallskip
\noindent
\textbf{In plain English, what is this command doing?}
\vspace{4em}


\vfill
\hrule
\smallskip
\noindent
{\footnotesize
  If you would like more practice with regular expressions, check out
  \href{http://regexone.com/}{this online tutor}. One tricky thing, there
  are many different ``dialects'' of regular expressions. Standard
  \texttt{grep} is very fast but trades off speed for limited features. The
  tutor teaches what \texttt{grep} calls ``Extended Regular Expressions''. For
  example, Lesson~6 teaches quantifiers, \texttt{grep .z\{2\}} will not work,
  but \texttt{grep -E .z\{2\}} will.
}

\newpage
\noindent
We have one more tool we need to learn about before with can get to the grand
finale, and that's \texttt{uniq}. First, run this command:
\begin{lstlisting}[keepspaces=true]
> for i in $(seq 20); do echo $(($RANDOM % 5)) >> /tmp/numbers; done
             |                |  |
             |                |  \- this is a built-in bash "variable"
             |                |     try just "echo $RANDOM", what happens?
             |                |
             |                \- $(( ... )) lets us do math in bash, in this
             |                   case we're doing a mod as the range of
             |                   $RANDOM is quite large (0 - 32767)
             |
             \- seq is a program to print a sequence of numbers. Try just
                typing "seq 10", "seq 10 20", and "seq 10 2 20"

                You can also use the built-in bash range idiom:
                > for i in {1..10}; do echo $i; done
                to accomplish the same task, but I find that harder to remember the syntax
\end{lstlisting}
%
Check out the contents of the file \texttt{/tmp/numbers}. Do you understand
what that command did?
\medskip
\noindent
Now try running \texttt{uniq /tmp/numbers}. What does the \texttt{uniq}
command do? Not sure? Try looking at the output side-by-side:\footnote{
  The \texttt{<( ...\ )} redirect lets you run a command and use its output for
  something that expects a file, similar to how \texttt{\$( ...\ )} lets you
  run a command and use its output as text. It lets you skip creating a
  temporary file. Otherwise you could run\\\texttt{
  > uniq /tmp/numbers > /tmp/uniq-output\\
  > diff -y /tmp/numbers /tmp/uniq-output\\
  }
  and the result would be the same.
}\\
\texttt{> diff -y /tmp/numbers <(uniq /tmp/numbers)}

\medskip
\noindent
The man page for \texttt{uniq} is short and simple. It is a good man page. Try
giving it a read and playing with some of the other options for \texttt{uniq}.


\newpage
\medskip
\noindent
\textbf{\large Now, for the hard part: Solving EECS\,281 problems in 100 characters or less.}\\
We're going to combine everything we've learned so far to answer some powerful
queries. For each question, you need to come up with a string of commands
stuck together with pipes that will return the answer to the question. The
correct answer is also given for each question so that you can check your
work.

\medskip
\noindent
\emph{\small Hint: While we often use the \texttt{cat} utility to just print
  the contents of a single file, it's real purpose is for con\textbf{cat}enating
  multiple files. How might concatenating files be useful in conjunction with
  these other utilities?
}

\medskip
\noindent
\emph{\small Hint: The way to approach this problem (and many problems) is to
  build it up from small pieces. String together two commands until they give
  you want you want, then add a third, etc.
}


\bigskip
\hrule
\smallskip
\noindent
\textbf{How many words are in only the \texttt{british-english} word list or
the \texttt{american-english} word list but are not in both?}
\hfill \textbf{[Answer: 3481 words not in common]}\\
\emph{\small If for some reason there is no \texttt{british-english} file, run
\texttt{sudo apt-get install wbritish}}
\vspace{3cm}

\hrule
\smallskip
\noindent
\textbf{How many entries in the easily crackable password list
  (\texttt{cracklib-small}) are English words (are in
\texttt{american-english})?}
\hfill \textbf{[Answer: 40599 words in common]}
\vspace{3cm}

\vfill
\section*{\ul{Optional} Related Readings}
Quick and light, I particularly recommend the first one. It's a good anecdote
for software and system design.

\medskip
\noindent
\href{http://www.leancrew.com/all-this/2011/12/more-shell-less-egg/}{More
shell, less egg} -- This is a fun blog story of when even
\href{https://en.wikipedia.org/wiki/Donald_Knuth}{Donald Knuth} sometimes gets
things wrong.
% n.b. this reading actually gives away the HW a little, but if people read
% this I think they'll get the benefit either way

\medskip
\noindent
\href{https://lists.freebsd.org/pipermail/freebsd-current/2010-August/019310.html}{``Why GNU grep is so fast''} --
This is a nice example of the importance of efficient algorithm design and how
an implementation can benefit from a deep understanding of the underlying
system.




\newpage

\section{Pulling Some Pieces Together}

\medskip
\noindent
Head back to the git repository you created for week 2's homework.

\medskip
\noindent
Week 2's homework had you blindly run the command
\begin{lstlisting}
> grep ';' p2.h | grep -v ' \*' >> p2.cpp
\end{lstlisting}
%
Now try running
\begin{lstlisting}
> grep ';' p2.h | grep -v ' \*'
> grep ';' p2.h | grep ' \*'
> grep ';' p2.h | grep -v ' \*' | grep filter
> grep ';' p2.h | grep -v ' \*' | grep -v filter
\end{lstlisting}
%
\textbf{In plain English, dissect the command \texttt{grep -v \textquotesingle~\textbackslash*\textquotesingle} (notice the space)}
\vspace{3cm}


\noindent
(Run \texttt{make} if you haven't already)

\medskip
\noindent
Suppose you wanted to change the \texttt{insert\_list} function, so you wanted
to find all of the places it's called. One approach would be:
\begin{lstlisting}
> grep insert_list *
\end{lstlisting}
which searches all files for the string ``insert\_list''. Compare that search,
however, to
\begin{lstlisting}
> git grep insert_list
\end{lstlisting}
Do you see how the two searches differ?

\medskip
\noindent
\textbf{Does `git grep' search untracked files? How do you know?}
\vspace{3cm}


\noindent
\textbf{Does `git grep' search new files that have been staged but not
committed? What test could you quickly run to find out?}
\vspace{3cm}

\end{document}
