\documentclass{article}
\usepackage[T1]{fontenc}

\usepackage{amssymb}
\usepackage{courier}
\usepackage{fancyhdr}
\usepackage{fancyvrb}
\usepackage[top=.75in, bottom=.75in, left=.75in,right=.75in]{geometry}
\usepackage{graphicx}
\usepackage{lastpage}
\usepackage{listings}
\lstset{basicstyle=\small\ttfamily}
\usepackage{mdframed}
\usepackage{parskip}
\usepackage{soul}
\usepackage{tabularx}
\usepackage{textcomp}
\usepackage{upquote}
\usepackage{xcolor}

% http://www.monperrus.net/martin/copy-pastable-ascii-characters-with-pdftex-pdflatex
\lstset{
upquote=true,
columns=fullflexible,
keepspaces=true,
literate={*}{{\char42}}1
         {-}{{\char45}}1
         {^}{{\char94}}1
}
\lstset{
  moredelim=**[is][\color{blue}\bf\small\ttfamily]{@}{@},
}

% http://tex.stackexchange.com/questions/40863/parskip-inserts-extra-space-after-floats-and-listings
\lstset{aboveskip=6pt plus 2pt minus 2pt, belowskip=-4pt plus 2pt minus 2pt}



\usepackage[colorlinks,urlcolor={blue}]{hyperref}
\usepackage[capitalise,nameinlink,noabbrev]{cleveref}
\crefname{section}{Question}{Questions}
\Crefname{section}{Question}{Questions}

\begin{document}

%\fancyhead[C]{\hl{Select the right page in Gradescope or we will not grade the question!}}
%\fancyhead[L]{}
%\fancyhead[R]{}

\fancyfoot[L]{\color{gray} C4CS -- W'17}
\fancyfoot[R]{\color{gray} Revision 1.0}
\fancyfoot[C]{\color{gray} \thepage~/~\pageref*{LastPage}}
\pagestyle{fancyplain}



\title{\textbf{Homework 12\\Towards a Development Environment}}
\author{\textbf{\color{red}{Due: Saturday, April 1, 10:00PM (Hard Deadline)}}}
\date{}
\maketitle


\section*{Submission Instructions}
Submit a pointer to your dotfiles repository at
\href{https://goo.gl/forms/3HH8KiQo8RHwu7zE2}{https://goo.gl/forms/3HH8KiQo8RHwu7zE2}.

\section{A dotfiles repository}

As we've seen in class, many tools store their configuration in ``dotfiles''
in your home directory. While this is convenient for configuring things on one
machine, often you may have multiple machines that you use (desktop at
work, laptop for travel, etc). It would be really nice if all of these things
stayed in sync.

Fortunately, we've also learned a great tool for keeping things synchronized
in this class -- git! Keeping your configuration files in git has the added
benefit that you can try changes that your aren't sure if you will like
(e.g\ trying out \href{http://blog.sanctum.geek.nz/vi-mode-in-bash/}{vi mode
in bash}) knowing that you can easily revert them later.

The trick is that it's not a good idea to make your home directory a git
repository (because then every file in your home directory would be in that
repository!). Instead, create a folder named \texttt{dotfiles}, make that a
repository, and then use symlinks to point from where tools expect your
dotfiles (in your home directory) to where they actually are (the dotfiles
folder):

\begin{lstlisting}
  $ ls -l ~/.bashrc
  lrwxr-xr-x  1 ppannuto  staff  40 Mar 31 23:43 /Users/ppannuto/.bashrc ->
  /Users/ppannuto/Dropbox/dotfiles/.bashrc
\end{lstlisting}


\section{Remote work and configuring connections}
Previously, we have directed you to look at \texttt{ssh} and \texttt{scp}
for doing work on and moving files to and from remote machines. Repeated
typing of anything in a Linux environment should be leading you to asking,
``Is there any way I can do this with less typing?'' The answer lies with the
\emph{OpenSSH SSH client configuration files}, whose reference can be found
with \texttt{man ssh\_config}. Full documentation can be found on the man
page, of course, but some more concrete examples and sample files are right
at the end of a google search.

This investigation will show you that hosts can be configured and referred to
using simple (short) nicknames defined in \textasciitilde/.ssh/config. An
example from the instructor's config file looks like:

\begin{lstlisting}
# UMich GitLab
Host gitlab
  HostName gitlab.com
  User git
  IdentityFile ~/.ssh/mmdarden_rsa

# CAEN
Host mmd
  HostName login.engin.umich.edu
  User mmdarden
  Compression yes
  
# Home media server
Host nas
  HostName 10.180.239.64
  User darden
  IdentityFile ~/.ssh/id_rsa
\end{lstlisting}

These hosts can be specified by their shorter nicknames anywhere you would
use the full User@HostName. This leads to simplified logins, shorter
\texttt{git clone} commmands, and easier file management with \texttt{scp},
among other things:

\begin{lstlisting}
  $ ssh mmd  # login to CAEN as mmdarden at login.engin.umich.edu
  $ git clone gitlab:mmdarden/vimconfig.git
  $ scp nas:music/artist/album/song.mp3 .  # copy a song to my local machine
\end{lstlisting}
  

\subsection*{The Assignment}
Create a dotfiles repository. Move your existing \texttt{.bashrc} file into
this repository, commit it, and set up a symlink from your home directory for
bashrc. Create an \texttt{ssh\_config} file and add it to your dotfiles
repository. Finally, pick at least one other dotfile to move under version
control and add it to this repository. For ideas, here are the files in my
dotfiles folder:

\begin{lstlisting}[escapechar=!]
  .bash_aliases .bash_profile .bashfuncs .bashrc .gitconfig .gitignore .gsdesktop-helper!\footnote{Pour one out... (dead config file from \href{https://github.com/intarstudents/GSDesktop-Helper}{this} for \href{https://en.wikipedia.org/wiki/Grooveshark}{Grooveshark})}!
  .hgrc .inputrc .pypirc .screenrc .vim!\footnote{This is actually a directory for \href{https://github.com/tpope/vim-pathogen}{Pathogen}}!  .vimrc
  setup_dotfiles.sh!\footnote{This is a script that deletes any existing dotfiles (e.g.\ things usually ship with a template \texttt{.bashrc}) and replaces them with symlinks to the dotfile repo. It also handles ssh-config, which lives at \textasciitilde/.ssh/config}!
  ssh-config
\end{lstlisting}

You can also find \emph{lots} of other dotfiles repositories on the web with
tons of cool ideas for configurations. Some of the staff even have pretty good
dotfiles repositories\dots

Use any repository hosting service (Umich Gitlab, GitHub, personal server,
etc) to make your dotfiles repo publically accessible. Then submit a link to
your repository at
\href{https://goo.gl/forms/3HH8KiQo8RHwu7zE2}{https://goo.gl/forms/3HH8KiQo8RHwu7zE2}.

\end{document}
