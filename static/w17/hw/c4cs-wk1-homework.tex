\documentclass{article}
\usepackage[T1]{fontenc}

\usepackage{amssymb}
\usepackage{courier}
\usepackage{fancyhdr}
\usepackage{fancyvrb}
\usepackage[top=.75in, bottom=.75in, left=.75in,right=.75in]{geometry}
\usepackage{graphicx}
\usepackage{lastpage}
\usepackage{listings}
\lstset{basicstyle=\small\ttfamily}
\usepackage{mdframed}
\usepackage{parskip}
\usepackage{soul}
\usepackage{tabularx}
\usepackage{textcomp}
\usepackage{upquote}
\usepackage{xcolor}

% http://www.monperrus.net/martin/copy-pastable-ascii-characters-with-pdftex-pdflatex
\lstset{
upquote=true,
columns=fullflexible,
keepspaces=true,
literate={*}{{\char42}}1
         {-}{{\char45}}1
         {^}{{\char94}}1
}
\lstset{
  moredelim=**[is][\color{blue}\bf\small\ttfamily]{@}{@},
}

% http://tex.stackexchange.com/questions/40863/parskip-inserts-extra-space-after-floats-and-listings
\lstset{aboveskip=6pt plus 2pt minus 2pt, belowskip=-4pt plus 2pt minus 2pt}



\usepackage[colorlinks,urlcolor={blue}]{hyperref}
\usepackage[capitalise,nameinlink,noabbrev]{cleveref}
\crefname{section}{Question}{Questions}
\Crefname{section}{Question}{Questions}

\begin{document}

\fancyhead[C]{\hl{Select the right page in Gradescope or we will not grade the question!}}
\fancyhead[L]{}
\fancyhead[R]{}

\fancyfoot[L]{C4CS -- W'17}
\fancyfoot[R]{Revision 1.0}
\fancyfoot[C]{\thepage~/~\pageref*{LastPage}}
\pagestyle{fancyplain}


\title{\textbf{Homework 1\\Welcome, Setup, and Some Light Reading}}
%\author{Assigned: Friday, January 8, 11:00AM}
\author{\textbf{\color{red}{Due: Saturday, January 14th, 10:00PM (Hard Deadline)}}}
\date{}
\maketitle


\section*{Submission Instructions}
Submit this assignment on \href{https://gradescope.com/courses/3499}{Gradescope}.
You may find the free online tool \href{https://www.pdfescape.com}{PDFescape}
helpful to edit and fill out this PDF.
You may also print, handwrite, and scan this assignment.


\section{Set Up a Ubuntu Virtual Machine}

One of the goals of this class is to understand systems work so that you can
customize and improve them for yourself.
On CAEN, course environments are already set up and everything ``just works''.
On a brand new Ubuntu install, however, we will have to find, install, set up
and manage many tools ourselves.


Recall from lecture that a virtual machine (VM) is a fake computer running as
a program. We'll use a VM in this course as a playground to test things out
and work without risking anything on your day-to-day machine. To kick things
off, we start by getting a basic environment set up this week.

\medskip
\noindent
\emph{One final thought:} Homework in this class will often be a little
underspecified. You are expected to Google, to try things, and to fail from
time to time. Making mistakes is highly encouraged, it's how you learn. We
have many office hours if you find yourself getting stuck, but we will always
start with the questions, ``What have you tried so far?'' and ``Why do you
think that didn't work?''

\medskip

\begin{enumerate}
  \item Get a copy of the \textbf{Desktop} version of \textbf{Ubuntu 16.04}
    (this is a big download, consider doing it on campus).
  \item Download and install \href{https://www.virtualbox.org}{VirtualBox}.
  \item Open VirtualBox and create a new virtual machine. Most of the defaults
    are fine. The default hard drive size of 8\,GB is a little small, I
    recommend going bigger (50\,GB or so). By default, disk images are
    \emph{sparse}, which means it won't take 50\,GB of real disk space to
    create a fake disk, rather the fake disk will grow on demand as it's used,
    so there's not a lot of harm in choosing a big number.
  \item Install Ubuntu on your new virtual machine. I recommend ``Downloading
    updates while installing''.
  \item Once Ubuntu is running, install the Guest Additions (try VirtualBox's
    Devices menu $\rightarrow$ Insert Guest Additions CD Image; you'll need to
    reboot once this finishes).
    \\
    \textbf{Q: What are Guest Additions? What do they do? What changed after
    you installed them and rebooted your VM?}
    \vspace{3.5cm}
  \item Play around with your new machine! Try writing and running a Hello
    World program. What about other tools you've used before? Can you get an
    old course project running? How is it different than a CAEN environment?
\end{enumerate}


\newpage
\section{Readings}

Each of these are short blog posts, 5-10~minute reads. I selected these to
give you a little exposure to some varying perspectives. The authors, Joel in
particular, have several other very interesting posts that I highly encourage
exploring. After each reading, write a response for the given question.

\medskip

\noindent
\textbf{Biculturalism} by Joel Spolsky\\
\url{http://www.joelonsoftware.com/articles/Biculturalism.html}\\

\medskip

\textbf{Q: Has your computing experience thus far aligned more with ``Windows
culture'' or ``unix culture''? What makes you feel that way?}

\vspace{7cm}

\noindent
\emph{These two articles use the word ``research'' a lot, but the points made
apply well to any work in computer science.}\\
~\\
\textbf{Helping my students overcome command-line bullshittery} by Phillip Guo\\
\url{http://www.pgbovine.net/command-line-bullshittery.htm}\\
and the counter-point\\
\textbf{On the value of command-line ``bullshittery''} by Eytan Adar\\
\url{https://medium.com/@eytanadar/on-the-value-of-command-line-bullshittery-94dc19ec8c61}\\

\medskip

\textbf{Q: What did you take away from these articles?}


\end{document}
