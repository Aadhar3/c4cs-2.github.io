\documentclass{article}
\usepackage{amssymb}
\usepackage{courier}
\usepackage{fancyhdr}
\usepackage{fancyvrb}
\usepackage[T1]{fontenc}
\usepackage[top=.75in, bottom=.75in, left=.75in,right=.75in]{geometry}
\usepackage{graphicx}
\usepackage{lastpage}
\usepackage{listings}
\lstset{basicstyle=\small\ttfamily}
\usepackage{mdframed}
\usepackage{parskip}
\usepackage{ragged2e}
\usepackage{soul}
\usepackage{upquote}
\usepackage{xcolor}

% http://www.monperrus.net/martin/copy-pastable-ascii-characters-with-pdftex-pdflatex
\lstset{
upquote=true,
columns=fullflexible,
literate={*}{{\char42}}1
         {-}{{\char45}}1
         {^}{{\char94}}1
}

% http://tex.stackexchange.com/questions/40863/parskip-inserts-extra-space-after-floats-and-listings
\lstset{aboveskip=6pt plus 2pt minus 2pt, belowskip=-4pt plus 2pt minus 2pt}

\usepackage[colorlinks,urlcolor={blue}]{hyperref}

\begin{document}

\fancyfoot[L]{\color{gray} C4CS -- W'17}
\fancyfoot[R]{\color{gray} Revision 1.0}
\fancyfoot[C]{\color{gray} \thepage~/~\pageref*{LastPage}}
\pagestyle{fancyplain}


\title{\textbf{Advanced Exercise -- Week 7\\}}
\author{\textbf{\color{red}{Due: Before March 10, 10:00PM}}}
\date{}
\maketitle


\section*{Submission Instructions}
To receive credit for this assignment you will need to stop by someone's
office hours, demo your running code, and answer some questions.

\begin{mdframed}[innerleftmargin=38pt,innerrightmargin=38pt]\justify
  This assignment asks to you work with any EECS project. You may use a
  current project, a previous project, or the EECS\,280~W15 project we've been
  using throughout the homeworks so far.

  If you do use a current EECS project, remember to commit everything before
  experimenting (you are using version control for all your projects, right?).
  Some of the examples reference the EECS\,280~W15 project, however they should
  be easy enough to translate to any other project.
\end{mdframed}

\section{Compiler-Assisted Makefiles}

As we saw in the regular homework, it can be somewhat hard to correctly list
all of the dependencies of a file. However, there is one program that knows
all of the files that the compiler needs to read in order to compile your
code, and that program is the compiler!

Try running \texttt{gcc~-MM~simple\_test.cpp} (also try running
\texttt{gcc~-M~simple\_test.cpp}, do you understand the difference?).
%One thing to note is that the target for the emitted rule is an object file
%(\texttt{simple\_test.o}), not the final executable (\texttt{simple\_test}).
What does \texttt{-MMD} do?

\textbf{Modify the makefile to auto-generate the dependencies for
  \texttt{simple\_test} and use them when building \texttt{simple\_test}.}

Notes / tips:
\begin{itemize}
  \item You \textbf{do not} need to set up the whole project to use
    auto-generated dependencies. This is tricky to do correctly.
    The goal is for you to play with this a little just to see how the
    mechanisms work. People have since developed tools that are easier to use
    and understand than gcc+make for auto-generating dependencies.\footnote{
      This is not to say that gcc's dependency generation is not useful.
      Indeed, \href{https://ninja-build.org/manual.html\#ref_headers}{ninja
        uses it}. Ninja, however explicitly understands that generating
      dependencies is a step in the build process, making it conceptually
      easier and cleaner than make.
    }
    \begin{itemize}
      \item If your solution is full of \texttt{\%}'s, you should be prepared
        to explain in depth how everything is working and what everything is
        doing.
    \end{itemize}
  \item Auto-generated dependencies generally add additional rules. Their
    purpose is to make sure that \emph{updates} work correctly. They are not
    needed for the first build. This means that to start, you only need to
    modify an existing rule that already builds code correctly to also build
    dependency information.
\end{itemize}

\subsection*{Submission checkoff:}
\begin{itemize}
  \item[$\square$] Explain the changes you made to your Makefile
    \begin{itemize}
      \item[$\square$] Show that auto-generated dependencies are working
        correctly (i.e. \texttt{make~\&\&~touch~recursive.h~\&\&~make}).
    \end{itemize}
\end{itemize}


\newpage
\section{Alternative Build Systems}

Make is a (relatively) simple program and comes installed on nearly every
system. However, as we've seen, make can be hard to use, and it can be hard to
get things completely correct. There are many other build systems out there.
This question asks you to try out my personal favorite and another one of your
own choosing.

\subsection{\href{http://gittup.org/tup/}{tup}}
The reason I like tup is that it aspires to be \emph{provably correct}. It
uses \href{https://en.wikipedia.org/wiki/Filesystem_in_Userspace}{FUSE} to
monitor every file system call, which means that if your compile command reads
or writes a file, tup knows about it. This also means that tup can
\emph{implicitly} do the same thing as gcc's \texttt{-M} options---the first
time the build runs it builds up a database of all the static files (mostly
header files) gcc reads and watches them for changes. Because the build system
is doing this, it works automatically and transparently for every command, not
just gcc.  For generated files this lets tup verify that your dependencies are
correct, e.g.\ if a rule attempts to run \texttt{./simple\_test}, a file built
by a different rule, but does not list \texttt{./simple\_test} as an input
dependency, tup will alert you with an error if you forgot to express a
dependency.

Tup also supports \href{http://gittup.org/tup/manual.html#lbAK}{variants},
which will automatically build multiple variations of your code.
%
As an example of why this is useful (likely near and dear to the heart of all
281 students), often it is useful to start and get things working with
optimizations turned off (\texttt{-Og})\footnote{
  Note, that's \texttt{-Og} \emph{not} \texttt{-O0}. Level~0 turns off
  \emph{all} optimizations and emits very slow and na\"ive code.
  \\
  From the gcc man page, ``Optimize debugging experience. -Og enables
  optimizations that do not interfere with debugging. It should be the
  optimization level of choice for the standard edit-compile-debug cycle,
  offering a reasonable level of optimization while maintaining fast
  compilation and a good debugging experience.''
} and then turn optimizations to max (\texttt{-O3~-flto~-DNDEBUG}) to get the
speediest programs. Unfortunately, you can sometimes introduce bugs\footnote{
  Or really useful warnings. Some warnings the compiler can only discover when
  optimizations are turned on.
} when doing debug builds and then not notice until the next time you do a
release build. Wouldn't it be great to just always build both?

\textbf{Create a copy of an EECS project and convert it to use tup.}

\textbf{Set up (at least) two variants (e.g. ``debug'' and ``release'' variants).}

\subsection*{Submission checkoff:}
\begin{itemize}
  \item[$\square$] Show off basic tup functionality and your Tupfiles
  \item[$\square$] Show off your variants and explain the differences between
    them, why did you choose those build options?
\end{itemize}

\subsection{Another build system of your choosing}
There are many to choose from, some of the more appropriate ones to consider
may be cmake, scons, gradle, or ninja.

\textbf{Choose another build system, read about it, and try to understand what
it does that is different, ``better'', or ``worse'' compared to \texttt{make} or
\texttt{tup}.}

You don't need to set up a whole project using this build system, though
playing with it a little may help deepen you understanding of what it provides
and how it's different.

\subsection*{Submission checkoff:}
\begin{itemize}
  \item[$\square$] Which build system did you learn about? Why that one?
  \item[$\square$] What type of projects is this build system ``best'' at
    (what was it designed for)?
    \begin{itemize}
      \item[$\square$] What does it do differently that makes it better for
        this application?
    \end{itemize}
\end{itemize}

\end{document}
