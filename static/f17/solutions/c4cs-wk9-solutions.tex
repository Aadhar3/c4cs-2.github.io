\documentclass{article}
\usepackage[T1]{fontenc}

\usepackage{amssymb}
\usepackage{courier}
\usepackage{fancyhdr}
\usepackage{fancyvrb}
\usepackage[top=.75in, bottom=.75in, left=.75in,right=.75in]{geometry}
\usepackage{graphicx}
\usepackage{lastpage}
\usepackage{listings}
\lstset{basicstyle=\small\ttfamily}
\usepackage{mdframed}
\usepackage{parskip}
\usepackage{soul}
\usepackage{tabularx}
\usepackage{textcomp}
\usepackage{upquote}
\usepackage{xcolor}

% http://www.monperrus.net/martin/copy-pastable-ascii-characters-with-pdftex-pdflatex
\lstset{
upquote=true,
columns=fullflexible,
keepspaces=true,
literate={*}{{\char42}}1
         {-}{{\char45}}1
         {^}{{\char94}}1
}
\lstset{
  moredelim=**[is][\color{blue}\bf\small\ttfamily]{@}{@},
}

% http://tex.stackexchange.com/questions/40863/parskip-inserts-extra-space-after-floats-and-listings
\lstset{aboveskip=6pt plus 2pt minus 2pt, belowskip=-4pt plus 2pt minus 2pt}



\usepackage[colorlinks,urlcolor={blue}]{hyperref}
\usepackage[capitalise,nameinlink,noabbrev]{cleveref}
\crefname{section}{Question}{Questions}
\Crefname{section}{Question}{Questions}

\begin{document}


\fancyhead[C]{\hl{Select the right page in Gradescope or we will not grade the question!}}
\fancyhead[L]{}
\fancyhead[R]{}

\fancyfoot[L]{\color{gray} C4CS -- F'17}
\fancyfoot[R]{\color{gray} Revision 1.1}
\fancyfoot[C]{\color{gray} \thepage~/~\pageref*{LastPage}}
\pagestyle{fancyplain}

\title{\textbf{Homework 9\\Debugging}}
\author{\textbf{\color{red}{Due: Wednesday, November 15th, 11:59PM (Hard Deadline)}}}
\date{}
\maketitle


\section*{Submission Instructions}
Submit this assignment on \href{https://gradescope.com/courses/9999}{Gradescope}.
You may find the free online tool \href{https://www.pdfescape.com}{PDFescape}
helpful to edit and fill out this PDF.
You may also print, handwrite, and scan this assignment.


There may multiple answers for each question. If you are unsure,
state your assumptions and your reasoning for why you think your answer
makes sense.


\newpage
\section{\texttt{Git}ting started}

Please \texttt{clone} the following repository:

\begin{lstlisting}
@>@ git clone https://github.com/c4cs/debugging-basics.git
\end{lstlisting}

This repository contains 4 branches: \texttt{master}, \texttt{gdb-debug-1}, \texttt{gdb-debug-2}, and \texttt{valgrind-debug}. Each part of this homework will take place on a different branch.

\section{Finding primes}

The repository you have cloned is an attempt at a program to find prime
numbers. It will search from 3 up to a maximum number input by the user.

The program intends to follow the structure:
\begin{lstlisting}
  o prompt user for upper bound
  o for N=3..upper bound:
    - check if N is prime
      * if no prime n between 1..sqrt(N) divides N, then prime
    - save whether N is prime for future loops to use
    - if N is prime, print N
\end{lstlisting}

\subsection{Debugging with \texttt{gdb}}

\emph{You may find it helpful to consult the
  \href{https://c4cs.github.io/lectures/f17/week11}{gdb lecture notes}
  or this
  \href{https://ccrma.stanford.edu/~jos/stkintro/Useful_commands_gdb.html}
  {quick command reference}.
}



First, make sure you are on the \texttt{gdb-debug-1} branch:
\begin{lstlisting}
@>@ git branch
  master
* gdb-debug-1
\end{lstlisting}

Next, build and run the supplied program:
\begin{lstlisting}
@>@ make
@>@ ./prime
Find all prime numbers between 3 and ?
10
Segmentation fault (core dumped)
@>@ gdb -q ./prime
Reading symbols from ./prime...done.
(gdb) run
Starting program: /media/sf_prime/prime 
Find all prime numbers between 3 and ?
10

Program received signal SIGSEGV, Segmentation fault.
0x000000000040068f in check_prime (k=3) at check.c:15
@15      if (is_prime[j] == 1)@
\end{lstlisting}

\textbf{Explain in what case(s) executing this line of code could cause a
  segmentation fault?}
\begin{quote}
  \color{violet}
  This question is asking only about this specific line of code, not all of
  the things that could possibly cause segmentation faults.

  This is an array access, which is a fancy way of dereferencing a pointer.
  Since we are in C, \texttt{is\_prime[j]} is the exact same as
  \texttt{*(is\_prime + j)}.

  This means that this line of code could cause a segmentation fault if the
  value \texttt{is\_prime + j} does not point to a valid memory address.
  %
  This is most likely to happen if \texttt{is\_prime} is not a valid pointer
  or \texttt{abs(j)} is an enormous value.

  \emph{Worth Noting: While a value of \texttt{j=-1} is very likely incorrect,
    it is unlikely to cause a segmentation fault. It will simply point to a
    different, nearby variable in the program's memory space. The fact that
    bugs like this do not cause crashes (at least not immediately) can make
    them hard to track down.}
\end{quote}

\textbf{What gdb commands could you run next to prove your hypothesis right or
  wrong?}
\begin{quote}
  \color{violet}
  Either \texttt{print is\_prime} or \texttt{ptype is\_prime} to learn the
  size of the array, then \texttt{print j} to learn whether \texttt{j} is
  valid.

  Asking gdb to \texttt{print is\_prime[j]} will directly show that the memory
  pointed to is inaccessible.
\end{quote}


\newpage
% Fun fact: This was a "real" bug, that is, when I was writing this assignment
% I made a mistake and used the debugger to track it down
Now checkout \texttt{gdb-debug-2}:
\begin{lstlisting}
@>@ git checkout gdb-debug-2
@>@ make
@>@ ./prime
Find all prime numbers between 3 and ?
10
3 is a prime
5 is a prime
7 is a prime
@9 is a prime@
\end{lstlisting}
\textbf{Copy your debugging session and add notes explaining your
  thought process as you track down why this program thinks 9 is a prime.}\\
\emph{\small Hint: It looks like things go well up until 9. A good place to
  start then may be to break in and observe how the code determines whether 9
  is a prime number.}

\begin{quote}
    \color{violet}
  \begin{lstlisting}
    $ gdb -q ./prime
    Reading symbols from ./prime...Reading symbols from [...] ...done.
    done.

    @Things go well up until checking if 9 is prime, so jump there@
    (gdb) break check_prime if k==9
    Breakpoint 1 at 0x100000e87: file check.c, line 13.
    (gdb) run
    Starting program: /private/tmp/debugging-basics/prime
    Find all prime numbers between 3 and ?
    10
    3 is a prime
    5 is a prime
    7 is a prime

    Breakpoint 1, check_prime (k=9) at check.c:13
    13    for (j=2; j*j <= k; j++) { @expect to run this loop for j=2 and j=3@

    @There's not a whole lot going on in this loop. As we move through it, the
    only variable that changes is j. (We can infer the values of things like
    is_prime[j] based on the path the code takes, though you could certainly
    print those values as well if you like)@
    (gdb) display j
    1: j = 0
    (gdb) step
    14      if (is_prime[j] == 1) @expect this to be true, as 2 is prime@
    1: j = 2
    (gdb) <enter> @repeats last command, in this case step@
    15        if (k % j == 0)  { @9 % 2 isn't 0@
    1: j = 2
    (gdb) <enter>
    19      j++; @okay, onto the next iteration of the loop@
    1: j = 2
    (gdb) <enter>
    13    for (j=2; j*j <= k; j++) { @starting off the loop for j=3@
    1: j = 3
    (gdb) <enter>
    23    is_prime[k] = 1; @wait, we just dropped out of the loop, and j is 4@
    1: j = 4
    (gdb)
    @Key thing to notice: The loop body never executed for j=3. Why? j was
    incremented twice, once at the end of the body of the for loop, and once
    by the declaration of the for loop. This double-increment is the bug.@
  \end{lstlisting}
\end{quote}

\vfill
\hrule
{\footnotesize
This homework is based off of a simplified and updated version of this gdb
tutorial:
\url{http://heather.cs.ucdavis.edu/~matloff/UnixAndC/CLanguage/Debug.html}
If you would like some more practice, or simply a slightly different
explanation of the material, this may be a good source.
}


\newpage
\section{Valgrind}

Valgrind is a different kind of debugging tool. It is essentially a type of
virtual machine (like a simpler VirtualBox). Valgrind actually recompiles
every assembly instruction to allow it to intercept and monitor all of the
hardware calls made by the child process. While this is very powerful and
gives valgrind a lot of insight, it is also very slow, which can be
problematic for debugging large pieces of software.

Valgrind also can be challenging when libraries (such as the STL, or Boost) do
funny things with memory that result in a large number of false warnings. We
will look at valgrind in more depth during the second week on debugging
tools. For today, let's just explore the basic functionality and check out the
kind of things that valgrind can catch that gdb can't.

First, we'll need to install valgrind
\begin{lstlisting}
@>@ sudo apt-get install valgrind
\end{lstlisting}

Now, checkout the valgrind branch
\begin{lstlisting}
@>@ git checkout valgrind-debug
\end{lstlisting}

Notice (perhaps \texttt{gitk~-{}-all}) that the valgrind branch builds on the
master branch, with all the compiler warnings turned on. It has also
\ul{cherry-pick}ed the fix ``Stop searching for primes once sqrt(k) is
reached'', that is the same commit object that managed to end up on two
different histories.  This is one example of how git can be both very
powerful, and very confusing.

Finally, this branch adds a commit that consolidates they two source code
files into one and gets rid of all of the global variables in the process.
Unfortunately, this refactoring may also have inroduced a bug.

We run valgrind very similarly to gdb:
\begin{lstlisting}
@>@ make
@>@ valgrind ./prime
==11959== Memcheck, a memory error detector
==11959== Copyright (C) 2002-2015, and GNU GPL'd, by Julian Seward et al.
==11959== Using Valgrind-3.11.0 and LibVEX; rerun with -h for copyright info
==11959== Command: ./prime
==11959== 
Find all prime numbers between 3 and ?
10
3 is a prime@
==11959== Conditional jump or move depends on uninitialised value(s)
==11959==    at 0x400683: check_prime (prime.c:32)
==11959==    by 0x400739: main (prime.c:52)@
==11959== 
5 is a prime
7 is a prime
==11959== 
==11959== HEAP SUMMARY:
==11959==     in use at exit: 0 bytes in 0 blocks
==11959==   total heap usage: 0 allocs, 0 frees, 0 bytes allocated
==11959== 
==11959== All heap blocks were freed -- no leaks are possible
==11959== 
==11959== For counts of detected and suppressed errors, rerun with: -v
==11959== Use --track-origins=yes to see where uninitialised values come from
==11959== ERROR SUMMARY: 3 errors from 1 contexts (suppressed: 0 from 0)
\end{lstlisting}
{\small (The numbers in the left column are the process ID. They will be
  different for every person)}

\newpage
\textbf{Notice that ``3 is a prime'' printed before this warning. Why was this
  warning not emitted when running \texttt{check\_prime(3,~\dots)}?}
\begin{quote}
  \color{violet}
  Valgrind detects errors at run time. For \texttt{check\_prime(3,~\dots)},
  the for loop check \texttt{j*j~$<=$~k} (\texttt{2*2~$<=$~3}) fails and
  the body of the loop is never entered. This means \texttt{is\_prime[j]}
  (\texttt{is\_prime[2]}) never runs, which means valgrind does not detect
  the error.

  The key takeaway here is that because of how valgrind works, it can only
  catch an error when it actually happens.
\end{quote}

These tools really get powerful when you combine them. We can run valgrind in
a way that lets gdb connect to it, and allows us to debug/inspect whenever
valgrind detects a problem:
\begin{lstlisting}
@>@ valgrind --vgdb=yes --vgdb-error=0 ./prime
\end{lstlisting}
In another terminal, follow the directions that valgrind prints to connect
gdb. In this case, valgrind has already `run' the program for you and inserted
a breakpoint at the very beginning of the program, we just need to `continue'
it. The program will run until valgrind encounters an issue, at which point
valgrind will automatically break for you.
\begin{lstlisting}
@(gdb)@ continue

valgrind terminal:
==23589== Conditional jump or move depends on uninitialised value(s)
==23589==    at 0x400623: check_prime (prime.c:32)
==23589==    by 0x4006C4: main (prime.c:52)
==23589== 
@==23589== (action on error) vgdb me ...  # this is where valgrind waits for gdb@

gdb terminal:
Program received signal SIGTRAP, Trace/breakpoint trap.
0x0000000000400623 in check_prime (k=5, is_prime=0xffefff910) at prime.c:32
@32      if (is_prime[j] == 1)   # this is the problematic line of code@
\end{lstlisting}

~

\textbf{What is the value of \texttt{j} at this point?}\\
\begin{quote}
  \color{violet}\tt
  (gdb) print j\\
  j = 2
\end{quote}

\textbf{What is the value of \texttt{is\_prime[j]} at the point?}\\
\begin{quote}
  \color{violet}\tt
  (gdb) print is\_prime[j]\\
  = -16777744 $\leftarrow$ This will be a garbage value and vary person to person
\end{quote}

\textbf{Use git to look at the most recent commit. What line of code was
  deleted that should not have been?}
\begin{quote}
    \color{violet}
  \begin{lstlisting}
$ git log -p

[... trim some output ...]

  int main() {
+       int is_prime[MAX_PRIMES];
+       int upper_bound;
        int N;

        printf("Find all prime numbers between 3 and ?\n");
        scanf("%d", &upper_bound);

@-       is_prime[2] = 1;@
-
        for (N = 3; N <= upper_bound; N += 2) {
-               check_prime(N);
+               check_prime(N, is_prime);
                if (is_prime[N])
                        printf("%d is a prime\n",N);
  }
  \end{lstlisting}
\end{quote}


\end{document}
