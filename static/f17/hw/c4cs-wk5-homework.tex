\documentclass{article}
\usepackage[T1]{fontenc}

\usepackage{amssymb}
\usepackage{courier}
\usepackage{fancyhdr}
\usepackage{fancyvrb}
\usepackage[top=.75in, bottom=.75in, left=.75in,right=.75in]{geometry}
\usepackage{graphicx}
\usepackage{lastpage}
\usepackage{listings}
\lstset{basicstyle=\small\ttfamily}
\usepackage{mdframed}
\usepackage{parskip}
\usepackage{soul}
\usepackage{tabularx}
\usepackage{textcomp}
\usepackage{upquote}
\usepackage{xcolor}

% http://www.monperrus.net/martin/copy-pastable-ascii-characters-with-pdftex-pdflatex
\lstset{
upquote=true,
columns=fullflexible,
keepspaces=true,
literate={*}{{\char42}}1
         {-}{{\char45}}1
         {^}{{\char94}}1
}
\lstset{
  moredelim=**[is][\color{blue}\bf\small\ttfamily]{@}{@},
}

% http://tex.stackexchange.com/questions/40863/parskip-inserts-extra-space-after-floats-and-listings
\lstset{aboveskip=6pt plus 2pt minus 2pt, belowskip=-4pt plus 2pt minus 2pt}



\usepackage[colorlinks,urlcolor={blue}]{hyperref}
\usepackage[capitalise,nameinlink,noabbrev]{cleveref}
\crefname{section}{Question}{Questions}
\Crefname{section}{Question}{Questions}

\begin{document}


\fancyhead[C]{\hl{Select the right page in Gradescope or we will not grade the question!}}
\fancyhead[L]{}
\fancyhead[R]{}

\fancyfoot[L]{\color{gray} C4CS -- F'17}
\fancyfoot[R]{\color{gray} Revision 1.0}
\fancyfoot[C]{\color{gray} \thepage~/~\pageref*{LastPage}}
\pagestyle{fancyplain}

\title{\textbf{Homework 5\\Git II}}
\author{\textbf{\color{red}{Due: Friday, October 20th, 11:59PM (Hard Deadline)}}}
\date{}
\maketitle


\section*{Submission Instructions}
For this assignment, all of your submissions will be to GitLab.


\section*{Optional Reading}

\emph{Git is a purely functional data structure}, by Philip Nilsson, from Jayway.

\url{http://www.jayway.com/2013/03/03/git-is-a-purely-functional-data-structure/}

I highly encourage reading this when you have some time to read the article
carefully and think deeply about the material.
This article presents an excellent way of thinking about git and how it
operates.



\section{Evaluating git usage}

Earlier this semester, we asked you to use git with at least one project. Now
you will set up that project to be shared with the course staff. Visit
\url{https://gitlab.eecs.umich.edu} and create a new project named
\textbf{exactly} \texttt{c4cs-f17-wk5}.
Be sure to create this project as
\textbf{\ul{Private}} (not Interal or Public).

Add this new repository as a remote (\texttt{git remote add \dots}) to your
existing project.

Push the project to this new remote.

In GitLab, grant \textbf{\ul{Reporter}} permission to \texttt{cgagnon} and
\texttt{mmdarden}.
(Choose ``Members'' from the drop-down list from the settings gear in the top
right to manage this permission).

We will run a test script, checking our access to everyone's repository on
Thursday.


\newpage
\section{Handling merge conflicts}

\subsection{Content Conflict}

Clone \url{https://gitlab.eecs.umich.edu/c4cs/c4cs-f17-conflict1.git}

This repository has a \texttt{master} branch and a \texttt{merge\_me} branch
that have diverged. Merge the \texttt{merge\_me} branch into \texttt{master},
resolving the conflict.

When you are done, running \texttt{bash test.sh} should print ``Success'' and
running \texttt{python main.py} should print something reasonable (if
factually inaccurate now, hooray!).

Create a new repository in your GitLab named \textbf{exactly}
\texttt{c4cs-f17-conflict1}.
Be sure to create this project as
\textbf{\ul{Private}} (not Interal or Public).

Push your changes to your new repository. (\emph{\small Note: This will be a
  different ``remote''})

In GitLab, grant \textbf{\ul{Reporter}} permission to \texttt{cgagnon} and
\texttt{mmdarden}.
(Choose ``Members'' from the drop-down list from the settings gear in the top
right to manage this permission).

We will run a test script, checking our access to everyone's repository on
Thursday night.

\subsection{File Path Conflict}

Repeat the same steps for
\url{https://gitlab.eecs.umich.edu/c4cs/c4cs-f17-conflict2.git}

Be sure to read over the commit history so that you are sure that the result
of your merge has the right data!

\end{document}
