\documentclass{article}
\usepackage{amssymb}
\usepackage{comment}
\usepackage{courier}
\usepackage{fancyhdr}
\usepackage{fancyvrb}
\usepackage[T1]{fontenc}
\usepackage[top=.75in, bottom=.75in, left=.75in,right=.75in]{geometry}
\usepackage{graphicx}
\usepackage{lastpage}
\usepackage{listings}
\lstset{basicstyle=\small\ttfamily}
\usepackage{mdframed}
\usepackage{parskip}
\usepackage{ragged2e}
\usepackage{soul}
\usepackage{upquote}
\usepackage{xcolor}

% http://www.monperrus.net/martin/copy-pastable-ascii-characters-with-pdftex-pdflatex
\lstset{
upquote=true,
columns=fullflexible,
literate={*}{{\char42}}1
         {-}{{\char45}}1
         {^}{{\char94}}1
}
\lstset{
  moredelim=**[is][\color{blue}\bf\small\ttfamily]{@}{@},
}

% http://tex.stackexchange.com/questions/40863/parskip-inserts-extra-space-after-floats-and-listings
\lstset{aboveskip=6pt plus 2pt minus 2pt, belowskip=-4pt plus 2pt minus 2pt}

\usepackage[colorlinks,urlcolor={blue}]{hyperref}

\begin{document}

\fancyfoot[L]{\color{gray} C4CS -- F'16}
\fancyfoot[R]{\color{gray} Revision 1.0}
\fancyfoot[C]{\color{gray} \thepage~/~\pageref*{LastPage}}
\pagestyle{fancyplain}


\title{\textbf{Advanced Exercise -- Week 7\\}}
\author{\textbf{\color{red}{Due: Before November 5, 10:00PM}}}
\date{}
\maketitle


\section*{Contributing to an open source project}

While version control is useful for your own projects and helps with group
class projects, what really makes it awesome it how easy it is to collaborate
and contribute to big projects.

\section*{Submission Instructions}

 - For stuff to c4cs has to finish w/in the two week window

 - For external stuff, has to have a substantial bit started before window

 - For ours, no need to swing by OH. For others, please do.

\begin{mdframed}\centering
A good first place to start is by reading over
\href{https://guides.github.com/activities/contributing-to-open-source/#contributing}{GitHub's
  guide to contributing to open source projects}.
\end{mdframed}

\newpage

\section*{One Option: Contributing to the class website}

\subsection*{GitHub Pages}

Notice how the course homepage is \texttt{c4cs.\ul{github.io}}? GitHub has a
really neat feature called \href{https://pages.github.com/}{GitHub Pages},
that will turn a repository into a website -- hosted for free.\footnote{%
  My \href{https://github.com/ppannuto/patpannuto.com}{personal website} is also
  \href{https://github.com/ppannuto/ppannuto.github.io}{hosted this way},
  though it does not use jekyll.
} What's really nice about this is that it means it is very easy for many
people to collaborate to develop a website, it's just a repository!

In the simplest setup, GitHub will simply serve the files in the repository as
static web pages. Writing lots of HTML by hand, however, can be a pain, so
GitHub supports a \emph{site generator}, \href{https://jekyllrb.com/}{jekyll}.
Using jekyll, adding an update to the homepage is as simple as adding
\href{https://github.com/c4cs/c4cs.github.io/blob/master/_updates/f16/2016-10-11-chaos.md}{a text file}.
The site will also
\href{https://github.com/c4cs/c4cs.github.io/blob/master/index.html#L32}%
{automatically hide updates older than one month}.

\subsection*{Getting Going}

First, install ruby: \texttt{sudo apt-get install ruby ruby-dev zlib1g-dev}

Then, install \href{http://bundler.io}{bundler}: \texttt{sudo gem install bundler}

Now grab a copy of course website repository
(\url{https://github.com/c4cs/c4cs.github.io}), and follow the directions
in the readme.

Once things are running, it should print out
\begin{quote}\tt
  Server address: http://127.0.0.1:4000/
\end{quote}

Visit \url{http://127.0.0.1:4000/} in your browser and you should see your own
local copy of the course website! Try making some changes to the site and then
refresh the page in your browser.

\subsection*{The Stuff to Work on:}

If you haven't already, check out \url{https://c4cs.github.io/reference}

The goal is to build up a quick reference for common commands, as well as some
examples of how to use them and any gotcha's they may have. The hope is that
it's easier for people who are learning (i.e.\ you) to write documentation
that's helpful for other people who are learning as opposed to course staff
who don't really remember well what was confusing when we were first learning.

Look through the reference and find something that could use improvement. Or,
add a new command that we've talked about in class or that you have other
experience with that's not in the reference yet.

To ensure that people aren't working on the same things, before you get
started, check to see if there's an
\href{https://github.com/c4cs/c4cs.github.io/issues}{open issue} for what you
want to work on. If no one's doing it yet, make a new issue so that others
know what you're working on.

\newpage

\section*{Another Option: Any project of your choosing}

 - snag list from old EC project

 - point to betty, translations

 - point to umich research projects

\end{document}
